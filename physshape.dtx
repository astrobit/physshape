%	\iffalse meta-comment
%
%	Copyright (C) 2020 by Brian W. Mulligan <bwmulligan@astronaos.com>
% -----------------------------------------------------------
%
% This file may be distributed and/or modified under the conditions of
% the LaTeX Project Public License, either version 1.3c of this license
% or (at your option) any later version. The latest version of this
% license is in:
%
% http://www.latex-project.org/lppl.txt
%
% and version 1.3c or later is part of all distributions of LaTeX
% version 2006/05/20 or later.
%
% \fi
%
% \iffalse
%<*driver>
\ProvidesFile{physshape.dtx}
%</driver>
%<package>\NeedsTeXFormat{LaTeX2e}[1994/06/01]
%<package> \ProvidesPackage{physshape}
%<*package>
    [2020/01/22 v1.0.1 Physics and astronomy shape package]
%</package>
%<package>\RequirePackage{tikz}
%<package>\ProcessOptions\relax
%<*driver>
\documentclass{ltxdoc}
\usepackage{float}
\usepackage{tikz}
\usepackage{xcolor}
\usepackage{mdframed}
\usepackage{physshape}
\usepackage[backref]{hyperref}
\EnableCrossrefs
\CodelineIndex
\RecordChanges
\OnlyDescription
\begin{document}
\DocInput{physshape.dtx}
\PrintChanges
\PrintIndex
\end{document}
%</driver>
% \fi
%
% \CheckSum{4138}
%
% \CharacterTable
%  {Upper-case    \A\B\C\D\E\F\G\H\I\J\K\L\M\N\O\P\Q\R\S\T\U\V\W\X\Y\Z
%   Lower-case    \a\b\c\d\e\f\g\h\i\j\k\l\m\n\o\p\q\r\s\t\u\v\w\x\y\z
%   Digits        \0\1\2\3\4\5\6\7\8\9
%   Exclamation   \!     Double quote  \"     Hash (number) \#
%   Dollar        \$     Percent       \%     Ampersand     \&
%   Acute accent  \'     Left paren    \(     Right paren   \)
%   Asterisk      \*     Plus          \+     Comma         \,
%   Minus         \-     Point         \.     Solidus       \/
%   Colon         \:     Semicolon     \;     Less than     \<
%   Equals        \=     Greater than  \>     Question mark \?
%   Commercial at \@     Left bracket  \[     Backslash     \\
%   Right bracket \]     Circumflex    \^     Underscore    \_
%   Grave accent  \`     Left brace    \{     Vertical bar  \|
%   Right brace   \}     Tilde         \~}
%
% \changes{v1.0}{2020/01/22}{Initial version}
% \changes{v1.0.1}{2020/01/22}{Added better indexing and hyperref}
%
% \GetFileInfo{physsymb.dtx}
% \def\fileversion{v1.0}
% \def\filedate{2020/01/22}
%
% \DoNotIndex{\newcommand,\newenvironment,\DeclareRobustCommand}
% \DoNotIndex{\pgfdeclareshape,\savedanchor,\newdimen,\pgfextractx}
% \DoNotIndex{\pgfpointorigin,\advance,\wd,dp,\ht,\pgfnodeparttextbox}
% \DoNotIndex{\pgfpoint,\anchor,\anchorborder,\@tempdimxa,\@tempdimya,\pgf@x}
% \DoNotIndex{\pgf@y,\pgfpointborderrectangle,\pgfpointborderellipse}
% \DoNotIndex{\backgroundpath,\forgroundpath,\pgfpathmoveto,\pgfpathlineto}
% \DoNotIndex{\pgfpatharc,\pgfpatharcto,\pgfpathsine,\pgfpathcssine,\colorlet}
% \DoNotIndex{\pgfshadepath,\pgfusepath,\pgfsetfillcolor,\pgfsetstrokecolor}
% \DoNotIndex{\pgfsetcolor,\pgftext,\pgfscope,\endpgfscope,\pgfpathclose}
% \DoNotIndex{\pgfsetlinewidth,\color,\tiny,\small,\footnotesize,\left,\right}
% \DoNotIndex{\textbf,\mathrm,\pgfextractx,\pgfextracty,\northpoint,\eastpoint}
% \DoNotIndex{\westpoint,\southpoint,\northwestpoint,\northeastpoint}
% \DoNotIndex{\southeastpoint,\southwestpoint,\textpoint,\nodeparts}
% \DoNotIndex{\foregroundpath,\backgroundpath,\ancx,\ancy,\node,\draw}
% \DoNotIndex{\pgfpathcosine,\pgfpathsine,\pgfpointpolar,\pgfpointadd}
% \DoNotIndex{\pgfsetfillopacity,\pgfpathcircle}
%
%
%\newcommand{\NSmacro}[2]{% \DescribeMacro{#1}%
% The |#1| shape is #2.%
% \begin{figure}[H]%
% \begin{center}\begin{mdframed}\begin{center}\begin{tikzpicture}%
% 	\node[black, line width=0.5mm,% 
%       label={[black] above:#1}, #1, %
%       scale=5.0] (n1) at (0,0) { };%
%   \fill[blue] (n1.north) circle (0.05cm) node[right] {north};%
%   \fill[blue] (n1.south) circle (0.05cm) node[left] {south};%
%   \fill[red] (n1.center) circle (0.05cm) node[right] {center};%
% \end{tikzpicture}\end{center}\end{mdframed}\end{center}%
% \caption{\label{fig:gfx:#1}The #1 shape}%
% \end{figure}%
% Figure \ref{fig:gfx:#1} shows the #1 shape with the available anchors. %
% A text anchor is also defined (not shown) but should not be used. Labels are 
% recommended instead of using the node text.% 
% \begin{mdframed}[backgroundcolor=orange!25]%
% {\small\texttt{\textbackslash node[black, line width=0.5mm,\\ \hphantom{XXXX}label=\{[black] above:#1\}, #1,\\ \hphantom{XXXX}scale=5.0] (n1) at (0,0) \{ \};\\ \textbackslash fill[blue] (n1.north) circle (0.05cm) node[right] {north};\\ \textbackslash fill[blue] (n1.south) circle (0.05cm) node[left] {south};\\ \textbackslash fill[red] (n1.center) circle (0.05cm) node[right] {center};}}%
% \end{mdframed}%%
%}
%\newcommand{\NEWSmacro}[2]{% \DescribeMacro{#1}%
% The |#1| shape is #2.%
% \begin{figure}[H]%
% \begin{center}\begin{mdframed}\begin{center}\begin{tikzpicture}%
% 	\node[black, line width=0.5mm,% 
%       label={[black] 45:#1}, #1, %
%       scale=5.0] (n1) at (0,0) { };%
%   \fill[blue] (n1.north) circle (0.05cm) node[above] {north};%
%   \fill[blue] (n1.south) circle (0.05cm) node[below] {south};%
%   \fill[blue] (n1.east) circle (0.05cm) node[right] {east};%
%   \fill[blue] (n1.west) circle (0.05cm) node[left] {west};%
%   \fill[red] (n1.center) circle (0.05cm) node[above] {center};%
% \end{tikzpicture}\end{center}\end{mdframed}\end{center}%
% \caption{\label{fig:gfx:#1}The #1 shape}%
% \end{figure}%
% Figure \ref{fig:gfx:#1} shows the #1 shape with the available anchors. %
% A text anchor is also defined (not shown) but should not be used. Labels are 
% recommended instead of using the node text.% 
% \begin{mdframed}[backgroundcolor=orange!25]%
% {\small\texttt{\textbackslash node[black, line width=0.5mm,\\ \hphantom{XXXX}label=\{[black] 45:#1\}, #1,\\ \hphantom{XXXX}scale=5.0] (n1) at (0,0) \{ \};\\ \textbackslash fill[blue] (n1.north) circle (0.05cm) node[above] {north};\\ \textbackslash fill[blue] (n1.south) circle (0.05cm) node[below] {south};\\ \textbackslash fill[blue] (n1.east) circle (0.05cm) node[right] {east};\\ \textbackslash fill[blue] (n1.west) circle (0.05cm) node[left] {west};\\ \textbackslash fill[red] (n1.center) circle (0.05cm) node[above] {center};}}%
% \end{mdframed}%%
%}
%\newcommand{\EWmacro}[2]{% \DescribeMacro{#1}%
% The |#1| shape is #2.%
% \begin{figure}[H]%
% \begin{center}\begin{mdframed}\begin{center}\begin{tikzpicture}%
% 	\node[black, line width=0.5mm,% 
%       label={[black] above:#1}, #1, %
%       scale=5.0] (n1) at (0,0) { };%
%   \fill[blue] (n1.east) circle (0.05cm) node[right] {east};%
%   \fill[blue] (n1.west) circle (0.05cm) node[left] {west};%
%   \fill[red] (n1.center) circle (0.05cm) node[below] {center};%
% \end{tikzpicture}\end{center}\end{mdframed}\end{center}%
% \caption{\label{fig:gfx:#1}The #1 shape}%
% \end{figure}%
% Figure \ref{fig:gfx:#1} shows the #1 shape with the available anchors. %
% A text anchor is also defined (not shown) but should not be used. Labels are 
% recommended instead of using the node text.% 
% \begin{mdframed}[backgroundcolor=orange!25]%
% {\small\texttt{\textbackslash node[black, line width=0.5mm,\\ \hphantom{XXXX}label=\{[black] above:#1\}, #1,\\ \hphantom{XXXX}scale=5.0] (n1) at (0,0) \{ \};\\ \textbackslash fill[blue] (n1.east) circle (0.05cm) node[right] {east};\\ \textbackslash fill[blue] (n1.west) circle (0.05cm) node[left] {west};\\ \textbackslash fill[red] (n1.center) circle (0.05cm) node[right] {center};}}%
% \end{mdframed}%%
%}%
% \title{The \textsf{physshape} package\thanks{This document corresponds to \textsf{physshape}~\fileversion, dated \filedate.}}
% \author{Brian W. Mulligan \\ \texttt{bwmulligan@astronaos.com}}
%
% \maketitle
%
% \section{Introduction}
%
% 

%
% This package consists of several tikz / pgf shapes to be used in physics and 
% astronomy classes or documents.
% The package developed out of a physics class that I was teaching wherein I 
% wanted to be able to use tikz to draw circuits in a relatively straghttforward
% and easy way.
% In part, I didn't like the appearance of the shapes in the Circuit library,
% and I also didn't find existing shapes for most of the circuit elements that I
%  have provided here, e.g. cells, transformer, etc.
% I've added additional, non-circuit symbols on an as-needed basis for use in my
% slides and handouts for students.
%
%
%
% \section{Shapes}
%
% In this section I will show and describe the shapes provided by this package. 
% It is separated into two sections: circuit symbols and other symbols.
%
% \subsection{Circuit Symbols}
% \subsubsection{Cells}
%
% \index{Circuit Symbols>Cells|usage(}
% \NSmacro{cellVU}{a cell oriented vertically with the positive terminal upward}
% \NSmacro{cellVD}{a cell oriented vertically with the positive terminal downward}
% \EWmacro{cellHL}{a cell oriented horizontally with the positive terminal to the left}
% \EWmacro{cellHR}{a cell oriented horizontally with the positive terminal to the right}
% \index{Circuit Symbols>Cells|usage)}
%
% \subsubsection{Capacitors}
%
% \index{Circuit Symbols>Capacitors|usage(}
% \NSmacro{capacitorV}{a capacitor oriented vertically}
% \EWmacro{capacitorH}{a capacitor oriented horizontally}
% \index{Circuit Symbols>Capacitors|usage)}
%
% \subsubsection{Resistors}
%
% \index{Circuit Symbols>Resistors|usage(}
% \NSmacro{resistorV}{a resistor oriented vertically}
% \EWmacro{resistorH}{a resistor oriented horizontally}
% \index{Circuit Symbols>Resistors|usage)}
%
% \subsubsection{Inductors}
%
% \index{Circuit Symbols>Inductors|usage(}
% \NSmacro{inductorV}{an inductor oriented vertically}
% \EWmacro{inductorH}{an inductor oriented horizontally}
% \index{Circuit Symbols>Inductors|usage)}
%
% \subsubsection{Sources}
%
% \index{Circuit Symbols>Sources|usage(}
% \index{Circuit Symbols>Sources>AC|usage(}
% \NEWSmacro{ACsource}{an AC source}
% \index{Circuit Symbols>Sources>AC|usage)}
% \index{Circuit Symbols>Sources>DC|usage(}
% \NEWSmacro{DCsource}{a DC source}
% \index{Circuit Symbols>Sources>DC|usage)}
% \index{Circuit Symbols>Sources|usage)}
%
% \subsubsection{Meters}
%
% \index{Circuit Symbols>Meters|usage(}
% \index{Circuit Symbols>Meters>Ammeter|usage(}
% \NEWSmacro{voltmeter}{a voltmeter}
% \index{Circuit Symbols>Meters>Ammeter|usage)}
% \index{Circuit Symbols>Meters>Voltmeter|usage(}
% \NEWSmacro{ammeter}{an ammeter}
% \index{Circuit Symbols>Meters>Voltmeter|usage)}
% \index{Circuit Symbols>Meters|usage)}
%
% \subsubsection{Transformers}
%

% \index{Circuit Symbols>Transformers|usage(}
%\DescribeMacro{transformerH}%
% The |transformerH| shape is drawing of a transsformer circuit symbol,
% oriented horizontally.
% \begin{figure}[H]%
% \begin{center}\begin{mdframed}\begin{center}\begin{tikzpicture}%
% 	\node[black, line width=0.5mm,% 
%       label={[black] above:transformerH}, transformerH, %
%       scale=5.0] (n1) at (0,0) { };%
%   \fill[blue] (n1.northeast) circle (0.05cm) node[right] {northeast};%
%   \fill[blue] (n1.northwest) circle (0.05cm) node[left] {northwest};%
%   \fill[blue] (n1.southeast) circle (0.05cm) node[right] {southeast};%
%   \fill[blue] (n1.southwest) circle (0.05cm) node[left] {southwest};%
%   \fill[red] (n1.center) circle (0.05cm) node[below] {center};%
% \end{tikzpicture}\end{center}\end{mdframed}\end{center}%
% \caption{\label{fig:gfx:transformerH}The transformerH shape}%
% \end{figure}%
% Figure \ref{fig:gfx:transformerH} shows the transformerH shape with the available anchors. %
% A text anchor is also defined at the center. 
% \begin{mdframed}[backgroundcolor=orange!25]%
% \begin{verbatim}
% 	\node[black, line width=0.5mm,
%       label={[black] above:transformerH}, transformerH, 
%       scale=5.0] (n1) at (0,0) { };
%   \fill[blue] (n1.northeast) circle (0.05cm) node[right] {northeast};
%   \fill[blue] (n1.northwest) circle (0.05cm) node[left] {northwest};
%   \fill[blue] (n1.southeast) circle (0.05cm) node[right] {southeast};
%   \fill[blue] (n1.southwest) circle (0.05cm) node[left] {southwest};
%   \fill[red] (n1.center) circle (0.05cm) node[right] {center};
% \end{verbatim}
% \end{mdframed}

%\DescribeMacro{transformerV}%
% The |transformerV| shape is drawing of a transsformer circuit symbol,
% oriented vertically.
% \begin{figure}[H]%
% \begin{center}\begin{mdframed}\begin{center}\begin{tikzpicture}%
% 	\node[black, line width=0.5mm,% 
%       label={[black] above:transformerV}, transformerV, %
%       scale=5.0] (n1) at (0,0) { };%
%   \fill[blue] (n1.northeast) circle (0.05cm) node[right] {northeast};%
%   \fill[blue] (n1.northwest) circle (0.05cm) node[left] {northwest};%
%   \fill[blue] (n1.southeast) circle (0.05cm) node[right] {southeast};%
%   \fill[blue] (n1.southwest) circle (0.05cm) node[left] {southwest};%
%   \fill[red] (n1.center) circle (0.05cm) node[right] {center};%
% \end{tikzpicture}\end{center}\end{mdframed}\end{center}%
% \caption{\label{fig:gfx:transformerV}The transformerV shape}%
% \end{figure}%
% Figure \ref{fig:gfx:transformerV} shows the transformerV shape with the available anchors. %
% A text anchor is also defined at the center. 
% \begin{mdframed}[backgroundcolor=orange!25]%
% \begin{verbatim}
% 	\node[black, line width=0.5mm,
%       label={[black] above:transformerV}, transformerV, 
%       scale=5.0] (n1) at (0,0) { };
%   \fill[blue] (n1.northeast) circle (0.05cm) node[right] {northeast};
%   \fill[blue] (n1.northwest) circle (0.05cm) node[left] {northwest};
%   \fill[blue] (n1.southeast) circle (0.05cm) node[right] {southeast};
%   \fill[blue] (n1.southwest) circle (0.05cm) node[left] {southwest};
%   \fill[red] (n1.center) circle (0.05cm) node[right] {center};
% \end{verbatim}
% \end{mdframed}
% \index{Circuit Symbols>Transformers|usage)}
%
% \subsection{Switches}
%
% \index{Circuit Symbols>Switches|usage(}
% \index{Circuit Symbols>Switches>SPST|usage(}

% \NSmacro{switchtwoV}{a single contact (SPST) switch oriented vertically}
% \EWmacro{switchtwoH}{a single contact (SPST) switch oriented horizontally}
% \index{Circuit Symbols>Switches>SPST|usage)}

% \index{Circuit Symbols>Switches>SPDT|usage(}
%\DescribeMacro{switchthreeVDr}%
% The |switchthreeVDr| shape is drawing of two contact (SPDT) switch oriented 
% vertically with the single contact upward and the swith to the right.
% \begin{figure}[H]%
% \begin{center}\begin{mdframed}\begin{center}\begin{tikzpicture}%
% 	\node[black, line width=0.5mm,% 
%       label={[black] 45:switchthreeVDr}, switchthreeVDr, %
%       scale=5.0] (n1) at (0,0) { };%
%   \fill[blue] (n1.southeast) circle (0.05cm) node[right] {southeast};%
%   \fill[blue] (n1.southwest) circle (0.05cm) node[left] {southwest};%
%   \fill[blue] (n1.north) circle (0.05cm) node[above] {north};%
%   \fill[red] (n1.center) circle (0.05cm) node[above] {center};%
% \end{tikzpicture}\end{center}\end{mdframed}\end{center}%
% \caption{\label{fig:gfx:switchthreeVDr}The switchthreeVDr shape}%
% \end{figure}%
% Figure \ref{fig:gfx:switchthreeVDr} shows the switchthreeVDr shape with the available anchors. %
% A text anchor is also defined at the center. 
% \begin{mdframed}[backgroundcolor=orange!25]%
% \begin{verbatim}
% 	\node[black, line width=0.5mm,
%       label={[black] 45:switchthreeVDr}, switchthreeVDr, %
%       scale=5.0] (n1) at (0,0) { };
%   \fill[blue] (n1.southeast) circle (0.05cm) node[right] {southeast};%
%   \fill[blue] (n1.southwest) circle (0.05cm) node[left] {southwest};%
%   \fill[blue] (n1.north) circle (0.05cm) node[above] {north};%
%   \fill[red] (n1.center) circle (0.05cm) node[above] {center};%
% \end{verbatim}
% \end{mdframed}

%\DescribeMacro{switchthreeVDl}%
% The |switchthreeVDr| shape is drawing of two contact (SPDT) switch oriented 
% vertically with the single contact upward and the swith to the left.
% \begin{figure}[H]%
% \begin{center}\begin{mdframed}\begin{center}\begin{tikzpicture}%
% 	\node[black, line width=0.5mm,% 
%       label={[black] 45:switchthreeVDl}, switchthreeVDl, %
%       scale=5.0] (n1) at (0,0) { };%
%   \fill[blue] (n1.southeast) circle (0.05cm) node[right] {southeast};%
%   \fill[blue] (n1.southwest) circle (0.05cm) node[left] {southwest};%
%   \fill[blue] (n1.north) circle (0.05cm) node[above] {north};%
%   \fill[red] (n1.center) circle (0.05cm) node[above] {center};%
% \end{tikzpicture}\end{center}\end{mdframed}\end{center}%
% \caption{\label{fig:gfx:switchthreeVDl}The switchthreeVDl shape}%
% \end{figure}%
% Figure \ref{fig:gfx:switchthreeVDl} shows the switchthreeVDr shape with the available anchors. %
% A text anchor is also defined at the center. 
% \begin{mdframed}[backgroundcolor=orange!25]%
% \begin{verbatim}
% 	\node[black, line width=0.5mm,
%       label={[black] 45:switchthreeVDl}, switchthreeVDl, %
%       scale=5.0] (n1) at (0,0) { };
%   \fill[blue] (n1.southeast) circle (0.05cm) node[right] {southeast};%
%   \fill[blue] (n1.southwest) circle (0.05cm) node[left] {southwest};%
%   \fill[blue] (n1.north) circle (0.05cm) node[above] {north};%
%   \fill[red] (n1.center) circle (0.05cm) node[above] {center};%
% \end{verbatim}
% \end{mdframed}


%\DescribeMacro{switchthreeVUr}%
% The |switchthreeVUr| shape is drawing of two contact (SPDT) switch oriented 
% vertically with the single contact downward and the switch to the right.
% \begin{figure}[H]%
% \begin{center}\begin{mdframed}\begin{center}\begin{tikzpicture}%
% 	\node[black, line width=0.5mm,% 
%       label={[black] -45:switchthreeVUr}, switchthreeVUr, %
%       scale=5.0] (n1) at (0,0) { };%
%   \fill[blue] (n1.northeast) circle (0.05cm) node[right] {northeast};%
%   \fill[blue] (n1.northwest) circle (0.05cm) node[left] {northwest};%
%   \fill[blue] (n1.south) circle (0.05cm) node[below] {south};%
%   \fill[red] (n1.center) circle (0.05cm) node[below] {center};%
% \end{tikzpicture}\end{center}\end{mdframed}\end{center}%
% \caption{\label{fig:gfx:switchthreeVUr}The switchthreeVUr shape}%
% \end{figure}%
% Figure \ref{fig:gfx:switchthreeVUr} shows the switchthreeVDr shape with the available anchors. %
% A text anchor is also defined at the center. 
% \begin{mdframed}[backgroundcolor=orange!25]%
% \begin{verbatim}
% 	\node[black, line width=0.5mm,
%       label={[black] -45:switchthreeVUr}, switchthreeVUr, %
%       scale=5.0] (n1) at (0,0) { };
%   \fill[blue] (n1.southeast) circle (0.05cm) node[right] {southeast};%
%   \fill[blue] (n1.southwest) circle (0.05cm) node[left] {southwest};%
%   \fill[blue] (n1.north) circle (0.05cm) node[below] {south};%
%   \fill[red] (n1.center) circle (0.05cm) node[above] {center};%
% \end{verbatim}
% \end{mdframed}

%\DescribeMacro{switchthreeVUl}%
% The |switchthreeVUl| shape is drawing of two contact (SPDT) switch oriented 
% vertically with the single contact downward and the switch to the left.
% \begin{figure}[H]%
% \begin{center}\begin{mdframed}\begin{center}\begin{tikzpicture}%
% 	\node[black, line width=0.5mm,% 
%       label={[black] -45:switchthreeVUl}, switchthreeVUl, %
%       scale=5.0] (n1) at (0,0) { };%
%   \fill[blue] (n1.northeast) circle (0.05cm) node[right] {northeast};%
%   \fill[blue] (n1.northwest) circle (0.05cm) node[left] {northwest};%
%   \fill[blue] (n1.south) circle (0.05cm) node[below] {south};%
%   \fill[red] (n1.center) circle (0.05cm) node[below] {center};%
% \end{tikzpicture}\end{center}\end{mdframed}\end{center}%
% \caption{\label{fig:gfx:switchthreeVUl}The switchthreeVUl shape}%
% \end{figure}%
% Figure \ref{fig:gfx:switchthreeVUl} shows the switchthreeVUl shape with the available anchors. %
% A text anchor is also defined at the center. 
% \begin{mdframed}[backgroundcolor=orange!25]%
% \begin{verbatim}
% 	\node[black, line width=0.5mm,
%       label={[black] -45:switchthreeVUl}, switchthreeVUl, %
%       scale=5.0] (n1) at (0,0) { };
%   \fill[blue] (n1.southeast) circle (0.05cm) node[right] {southeast};%
%   \fill[blue] (n1.southwest) circle (0.05cm) node[left] {southwest};%
%   \fill[blue] (n1.north) circle (0.05cm) node[below] {south};%
%   \fill[red] (n1.center) circle (0.05cm) node[above] {center};%
% \end{verbatim}
% \end{mdframed}

%\DescribeMacro{switchthreeHLu}%
% The |switchthreeHLu| shape is drawing of two contact (SPDT) switch oriented 
% horizontally with the single contact leftward and the switch to upward.
% \begin{figure}[H]%
% \begin{center}\begin{mdframed}\begin{center}\begin{tikzpicture}%
% 	\node[black, line width=0.5mm,% 
%       label={[black] -45:switchthreeHLu}, switchthreeHLu, %
%       scale=5.0] (n1) at (0,0) { };%
%   \fill[blue] (n1.northwest) circle (0.05cm) node[left] {northwest};%
%   \fill[blue] (n1.southwest) circle (0.05cm) node[left] {southwest};%
%   \fill[blue] (n1.east) circle (0.05cm) node[right] {east};%
%   \fill[red] (n1.center) circle (0.05cm) node[left] {center};%
% \end{tikzpicture}\end{center}\end{mdframed}\end{center}%
% \caption{\label{fig:gfx:switchthreeHLu}The switchthreeHLu shape}%
% \end{figure}%
% Figure \ref{fig:gfx:switchthreeHLu} shows the switchthreeHLu shape with the available anchors. %
% A text anchor is also defined at the center. 
% \begin{mdframed}[backgroundcolor=orange!25]%
% \begin{verbatim}
% 	\node[black, line width=0.5mm,
%       label={[black] -45:switchthreeHLu}, switchthreeHLu, %
%       scale=5.0] (n1) at (0,0) { };
%   \fill[blue] (n1.southeast) circle (0.05cm) node[right] {southeast};%
%   \fill[blue] (n1.southwest) circle (0.05cm) node[left] {southwest};%
%   \fill[blue] (n1.north) circle (0.05cm) node[below] {south};%
%   \fill[red] (n1.center) circle (0.05cm) node[above] {center};%
% \end{verbatim}
% \end{mdframed}

%\DescribeMacro{switchthreeHLd}%
% The |switchthreeHLd| shape is drawing of two contact (SPDT) switch oriented 
% horizontally with the single contact leftward and the switch to upward.
% \begin{figure}[H]%
% \begin{center}\begin{mdframed}\begin{center}\begin{tikzpicture}%
% 	\node[black, line width=0.5mm,% 
%       label={[black] -45:switchthreeHLd}, switchthreeHLd, %
%       scale=5.0] (n1) at (0,0) { };%
%   \fill[blue] (n1.northwest) circle (0.05cm) node[left] {northwest};%
%   \fill[blue] (n1.southwest) circle (0.05cm) node[left] {southwest};%
%   \fill[blue] (n1.east) circle (0.05cm) node[right] {east};%
%   \fill[red] (n1.center) circle (0.05cm) node[left] {center};%
% \end{tikzpicture}\end{center}\end{mdframed}\end{center}%
% \caption{\label{fig:gfx:switchthreeHLd}The switchthreeHLd shape}%
% \end{figure}%
% Figure \ref{fig:gfx:switchthreeHLd} shows the switchthreeHLd shape with the available anchors. %
% A text anchor is also defined at the center. 
% \begin{mdframed}[backgroundcolor=orange!25]%
% \begin{verbatim}
% 	\node[black, line width=0.5mm,
%       label={[black] -45:switchthreeHLd}, switchthreeHLd, %
%       scale=5.0] (n1) at (0,0) { };
%   \fill[blue] (n1.northwest) circle (0.05cm) node[left] {northwest};%
%   \fill[blue] (n1.southwest) circle (0.05cm) node[left] {southwest};%
%   \fill[blue] (n1.east) circle (0.05cm) node[right] {east};%
%   \fill[red] (n1.center) circle (0.05cm) node[left] {center};%
% \end{verbatim}
% \end{mdframed}

%\DescribeMacro{switchthreeHRu}%
% The |switchthreeHRu| shape is drawing of two contact (SPDT) switch oriented 
% horizontally with the single contact leftward and the switch to upward.
% \begin{figure}[H]%
% \begin{center}\begin{mdframed}\begin{center}\begin{tikzpicture}%
% 	\node[black, line width=0.5mm,% 
%       label={[black] -45:switchthreeHRu}, switchthreeHRu, %
%       scale=5.0] (n1) at (0,0) { };%
%   \fill[blue] (n1.northeast) circle (0.05cm) node[right] {northeast};%
%   \fill[blue] (n1.southeast) circle (0.05cm) node[right] {southeast};%
%   \fill[blue] (n1.west) circle (0.05cm) node[left] {west};%
%   \fill[red] (n1.center) circle (0.05cm) node[right] {center};%
% \end{tikzpicture}\end{center}\end{mdframed}\end{center}%
% \caption{\label{fig:gfx:switchthreeHRu}The switchthreeHRu shape}%
% \end{figure}%
% Figure \ref{fig:gfx:switchthreeHRu} shows the switchthreeHRu shape with the available anchors. %
% A text anchor is also defined at the center. 
% \begin{mdframed}[backgroundcolor=orange!25]%
% \begin{verbatim}
% 	\node[black, line width=0.5mm,
%       label={[black] -45:switchthreeHRu}, switchthreeHRu, %
%       scale=5.0] (n1) at (0,0) { };
%   \fill[blue] (n1.northeast) circle (0.05cm) node[right] {northeast};%
%   \fill[blue] (n1.southeast) circle (0.05cm) node[right] {southeast};%
%   \fill[blue] (n1.west) circle (0.05cm) node[left] {west};%
%   \fill[red] (n1.center) circle (0.05cm) node[right] {center};%
% \end{verbatim}
% \end{mdframed}

%\DescribeMacro{switchthreeHRd}%
% The |switchthreeHRd| shape is drawing of two contact (SPDT) switch oriented 
% horizontally with the single contact leftward and the switch to upward.
% \begin{figure}[H]%
% \begin{center}\begin{mdframed}\begin{center}\begin{tikzpicture}%
% 	\node[black, line width=0.5mm,% 
%       label={[black] -45:switchthreeHRd}, switchthreeHRd, %
%       scale=5.0] (n1) at (0,0) { };%
%   \fill[blue] (n1.northeast) circle (0.05cm) node[right] {northeast};%
%   \fill[blue] (n1.southeast) circle (0.05cm) node[right] {southeast};%
%   \fill[blue] (n1.west) circle (0.05cm) node[left] {west};%
%   \fill[red] (n1.center) circle (0.05cm) node[right] {center};%
% \end{tikzpicture}\end{center}\end{mdframed}\end{center}%
% \caption{\label{fig:gfx:switchthreeHRd}The switchthreeHRd shape}%
% \end{figure}%
% Figure \ref{fig:gfx:switchthreeHRd} shows the switchthreeHRd shape with the available anchors. %
% A text anchor is also defined at the center. 
% \begin{mdframed}[backgroundcolor=orange!25]%
% \begin{verbatim}
% 	\node[black, line width=0.5mm,
%       label={[black] -45:switchthreeHRd}, switchthreeHRd, %
%       scale=5.0] (n1) at (0,0) { };
%   \fill[blue] (n1.northeast) circle (0.05cm) node[right] {northeast};%
%   \fill[blue] (n1.southeast) circle (0.05cm) node[right] {southeast};%
%   \fill[blue] (n1.west) circle (0.05cm) node[left] {west};%
%   \fill[red] (n1.center) circle (0.05cm) node[right] {center};%
% \end{verbatim}
% \end{mdframed}
% \index{Circuit Symbols>Switches>SPDT|usage)}
% \index{Circuit Symbols>Switches|usage)}


%
% \subsection{Other Symbols}
%
% \index{Proton|usage(}
%\DescribeMacro{proton}%
% The |sun| shape is drawing of a proton.
% \begin{figure}[H]%
% \begin{center}\begin{mdframed}\begin{center}\begin{tikzpicture}%
% 	\node[label={[black] above:proton}, proton, %
%       scale=1.0] (n1) at (0,0) { };%
%   \fill[red] (n1.center) circle (0.05cm) node[left] {center};%
% \end{tikzpicture}\end{center}\end{mdframed}\end{center}%
% \caption{\label{fig:gfx:proton}The proton shape}%
% \end{figure}%
% Figure \ref{fig:gfx:proton} shows the proton shape with the available anchors. %
% A text anchor is also defined at the center. 
% \begin{mdframed}[backgroundcolor=orange!25]%
% \begin{verbatim}
% 	\node[label={[black] above:proton}, proton, %
%       scale=5.0] (n1) at (0,0) { };%
%   \fill[red] (n1.center) circle (0.05cm) node[left] {center};%
% \end{verbatim}
% \end{mdframed}
% \index{Proton|usage)}

% \index{Electron|usage(}
%\DescribeMacro{electron}%
% The |sun| shape is drawing of a electron.
% \begin{figure}[H]%
% \begin{center}\begin{mdframed}\begin{center}\begin{tikzpicture}%
% 	\node[label={[black] above:electron}, electron, %
%       scale=1.0] (n1) at (0,0) { };%
%   \fill[red] (n1.center) circle (0.05cm) node[left] {center};%
% \end{tikzpicture}\end{center}\end{mdframed}\end{center}%
% \caption{\label{fig:gfx:electron}The electron shape}%
% \end{figure}%
% Figure \ref{fig:gfx:electron} shows the electron shape with the available anchors. %
% A text anchor is also defined at the center. 
% \begin{mdframed}[backgroundcolor=orange!25]%
% \begin{verbatim}
% 	\node[label={[black] above:electron}, electron, %
%       scale=5.0] (n1) at (0,0) { };%
%   \fill[red] (n1.center) circle (0.05cm) node[left] {center};%
% \end{verbatim}
% \end{mdframed}
% \index{Electron|usage)}

% \index{Cannon|usage(}
%\DescribeMacro{cannon}%
% The |sun| shape is drawing of a cannon.
% \begin{figure}[H]%
% \begin{center}\begin{mdframed}\begin{center}\begin{tikzpicture}%
% 	\node[label={[black] above:cannon}, cannon, %
%       scale=1.0] (n1) at (0,0) { };%
%   \fill[red] (n1.center) circle (0.05cm) node[left] {center};%
% \end{tikzpicture}\end{center}\end{mdframed}\end{center}%
% \caption{\label{fig:gfx:cannon}The cannon shape}%
% \end{figure}%
% Figure \ref{fig:gfx:cannon} shows the cannon shape with the available anchors. %
% A text anchor is also defined at the center. 
% \begin{mdframed}[backgroundcolor=orange!25]%
% \begin{verbatim}
% 	\node[label={[black] above:cannon}, cannon, %
%       scale=5.0] (n1) at (0,0) { };%
%   \fill[red] (n1.center) circle (0.05cm) node[left] {center};%
% \end{verbatim}
% \end{mdframed}
% \index{Cannon|usage)}


% \index{Eyeball|usage(}
%\DescribeMacro{eyeballleft}%
% The |sun| shape is drawing of an eye.
% \begin{figure}[H]%
% \begin{center}\begin{mdframed}\begin{center}\begin{tikzpicture}%
% 	\node[blue, line width=0.5mm,% 
%       label={[black] above:eyeballleft}, eyeballleft, %
%       scale=5.0] (n1) at (0,0) { };%
%   \fill[red] (n1.center) circle (0.05cm) node[left] {center};%
% \end{tikzpicture}\end{center}\end{mdframed}\end{center}%
% \caption{\label{fig:gfx:eyeballleft}The leftward eyeball shape with a blue iris}%
% \end{figure}%
% Figure \ref{fig:gfx:eyeballleft} shows the eyeballleft shape with the available anchors. %
% A text anchor is also defined at the center. 
% \begin{mdframed}[backgroundcolor=orange!25]%
% \begin{verbatim}
% 	\node[blue, line width=0.5mm,% 
%       label={[black] above:eyeballleft}, eyeballleft, %
%       scale=5.0] (n1) at (0,0) { };%
%   \fill[red] (n1.center) circle (0.05cm) node[left] {center};%
% \end{verbatim}
% \end{mdframed}
% \index{Eyeball|usage)}

% \index{Sun|usage(}
%\DescribeMacro{sun}%
% The |sun| shape is drawing of a star shape.
% \begin{figure}[H]%
% \begin{center}\begin{mdframed}\begin{center}\begin{tikzpicture}%
% 	\node[black, line width=0.5mm,% 
%       label={[black] above:sun}, sun, %
%       scale=5.0] (n1) at (0,0) { };%
%   \fill[blue] (n1.north east) circle (0.05cm) node[above right] {north east};%
%   \fill[blue] (n1.north west) circle (0.05cm) node[above left] {north west};%
%   \fill[blue] (n1.south east) circle (0.05cm) node[above right] {south east};%
%   \fill[blue] (n1.south west) circle (0.05cm) node[above left] {south west};%
%   \fill[blue] (n1.north) circle (0.05cm) node[below] {north};%
%   \fill[blue] (n1.south) circle (0.05cm) node[above] {south};%
%   \fill[blue] (n1.east) circle (0.05cm) node[right] {east};%
%   \fill[blue] (n1.west) circle (0.05cm) node[left] {west};%
%   \fill[red] (n1.center) circle (0.05cm) node[right] {center};%
% \end{tikzpicture}\end{center}\end{mdframed}\end{center}%
% \caption{\label{fig:gfx:sun}The sun shape}%
% \end{figure}%
% Figure \ref{fig:gfx:sun} shows the sun shape with the available anchors. %
% A text anchor is also defined at the center. 
% \begin{mdframed}[backgroundcolor=orange!25]%
% \begin{verbatim}
% 	\node[black, line width=0.5mm,% 
%       label={[black] above:sun}, sun, %
%       scale=5.0] (n1) at (0,0) { };%
%   \fill[blue] (n1.north east) circle (0.05cm)
%       node[above right] {north east};%
%   \fill[blue] (n1.north west) circle (0.05cm)
%       node[above left] {north west};%
%   \fill[blue] (n1.south east) circle (0.05cm)
%       node[above right] {south east};%
%   \fill[blue] (n1.south west) circle (0.05cm)
%       node[above left] {south west};%
%   \fill[blue] (n1.north) circle (0.05cm) node[below] {north};%
%   \fill[blue] (n1.south) circle (0.05cm) node[above] {south};%
%   \fill[blue] (n1.east) circle (0.05cm) node[right] {east};%
%   \fill[blue] (n1.west) circle (0.05cm) node[left] {west};%
%   \fill[red] (n1.center) circle (0.05cm) node[right] {center};%
% \end{verbatim}
% \end{mdframed}
% \index{Sun|usage)}
% \StopEventually{}
%
% \section{Implementation}
%
\makeatletter
%
% \begin{macro}{cellVU}
%
%
%%%%%%%%%%%%%%%%%%%%%%%%%%%%%%%%%%%%%%%%%%%%%%%%%%%%%%%%%%%%%%%%%%%%%%%%%%%%%%%%%%%%%%%%%%%%%%%%%%%%%%%%%%%%%%%%%%%%%%%%%%%%%%%%%%%%%%%%%%%%%%%%%%%%%%%%%%%%%%%%%%%%%
%%
%% Shape: cellVU
%%
%%%%%%%%%%%%%%%%%%%%%%%%%%%%%%%%%%%%%%%%%%%%%%%%%%%%%%%%%%%%%%%%%%%%%%%%%%%%%%%%%%%%%%%%%%%%%%%%%%%%%%%%%%%%%%%%%%%%%%%%%%%%%%%%%%%%%%%%%%%%%%%%%%%%%%%%%%%%%%%%%%%%%
%
%
%
% The |cellVU| shape is a cell with vertical orientation and + terminal upward
%
%    \begin{macrocode}
\pgfdeclareshape{cellVU}{
\savedanchor\textpoint{%
\newdimen\ancx
\newdimen\ancy
\pgfextractx{\ancx}{\pgfpointorigin}%
\pgfextracty{\ancy}{\pgfpointorigin}%
\advance\ancx by 0.35cm%
\advance\ancy by -.5\ht\pgfnodeparttextbox%
\pgfpoint{\ancx}{\ancy}%
}
\savedanchor\northpoint{%
\newdimen\ancx%
\newdimen\ancy%
\pgfextractx{\ancx}{\pgfpointorigin}%
\pgfextracty{\ancy}{\pgfpointorigin}%
\advance\ancy by 0.1cm%
\pgfpoint{\ancx}{\ancy}%
}
\savedanchor\southpoint{%
\newdimen\ancx%
\newdimen\ancy%
\pgfextractx{\ancx}{\pgfpointorigin}%
\pgfextracty{\ancy}{\pgfpointorigin}%
\advance\ancy by -0.1cm%
\pgfpoint{\ancx}{\ancy}%
}
\anchor{center}{\pgfpointorigin}
\anchor{north}{\northpoint}
\anchor{south}{\southpoint}
\anchor{text}{\textpoint}
\anchorborder{
\newdimen\@tempdimxa%
\newdimen\@tempdimya%
\@tempdimxa=\pgf@x
\@tempdimya=\pgf@y
\pgfpointborderrectangle{\pgfpoint{\@tempdimxa}{\@tempdimya}}
{\pgfpoint{0.3cm}{0.10cm}}
}
\backgroundpath{
%%
\pgfpathmoveto{\pgfpoint{-0.2cm}{-0.05cm}}
\pgfpathlineto{\pgfpoint{0.2cm}{-0.05cm}}
%%
\pgfpathmoveto{\pgfpoint{-0.3cm}{0.05cm}}
\pgfpathlineto{\pgfpoint{0.3cm}{0.05cm}}
%%
\pgfpathmoveto{\pgfpoint{0.0cm}{-0.05cm}}
\pgfpathlineto{\pgfpoint{0.0cm}{-0.10cm}}
%%
\pgfpathmoveto{\pgfpoint{0.0cm}{0.05cm}}
\pgfpathlineto{\pgfpoint{0.0cm}{0.10cm}}
\pgfusepath{stroke}
}
}
%    \end{macrocode}
% \end{macro}
%
% \begin{macro}{cellVD}
%
%
%%%%%%%%%%%%%%%%%%%%%%%%%%%%%%%%%%%%%%%%%%%%%%%%%%%%%%%%%%%%%%%%%%%%%%%%%%%%%%%%%%%%%%%%%%%%%%%%%%%%%%%%%%%%%%%%%%%%%%%%%%%%%%%%%%%%%%%%%%%%%%%%%%%%%%%%%%%%%%%%%%%%%
%%
%% Shape: Cell (VD)
%%
%%%%%%%%%%%%%%%%%%%%%%%%%%%%%%%%%%%%%%%%%%%%%%%%%%%%%%%%%%%%%%%%%%%%%%%%%%%%%%%%%%%%%%%%%%%%%%%%%%%%%%%%%%%%%%%%%%%%%%%%%%%%%%%%%%%%%%%%%%%%%%%%%%%%%%%%%%%%%%%%%%%%%
%
%
%
% The |cellVD| shape is a cell with vertical orientation and + terminal downward
%
%    \begin{macrocode}
\pgfdeclareshape{cellVD}{
\savedanchor\textpoint{%
\newdimen\ancx
\newdimen\ancy
\pgfextractx{\ancx}{\pgfpointorigin}%
\pgfextracty{\ancy}{\pgfpointorigin}%
\advance\ancx by 0.35cm%
\advance\ancy by -.5\ht\pgfnodeparttextbox%
\pgfpoint{\ancx}{\ancy}%
}
\savedanchor\northpoint{%
\newdimen\ancx%
\newdimen\ancy%
\pgfextractx{\ancx}{\pgfpointorigin}%
\pgfextracty{\ancy}{\pgfpointorigin}%
\advance\ancy by 0.1cm%
\pgfpoint{\ancx}{\ancy}%
}
\savedanchor\southpoint{%
\newdimen\ancx%
\newdimen\ancy%
\pgfextractx{\ancx}{\pgfpointorigin}%
\pgfextracty{\ancy}{\pgfpointorigin}%
\advance\ancy by -0.1cm%
\pgfpoint{\ancx}{\ancy}%
}
\anchor{center}{\pgfpointorigin}
\anchor{north}{\northpoint}
\anchor{south}{\southpoint}
\anchor{text}{\textpoint}
\anchorborder{
\newdimen\@tempdimxa%
\newdimen\@tempdimya%
\@tempdimxa=\pgf@x
\@tempdimya=\pgf@y
\pgfpointborderrectangle{\pgfpoint{\@tempdimxa}{\@tempdimya}}
{\pgfpoint{0.3cm}{0.10cm}}
}
\backgroundpath{
%%
\pgfpathmoveto{\pgfpoint{-0.3cm}{-0.05cm}}
\pgfpathlineto{\pgfpoint{0.3cm}{-0.05cm}}
%%
\pgfpathmoveto{\pgfpoint{-0.2cm}{0.05cm}}
\pgfpathlineto{\pgfpoint{0.2cm}{0.05cm}}
%%
\pgfpathmoveto{\pgfpoint{0.0cm}{-0.05cm}}
\pgfpathlineto{\pgfpoint{0.0cm}{-0.10cm}}
%%
\pgfpathmoveto{\pgfpoint{0.0cm}{0.05cm}}
\pgfpathlineto{\pgfpoint{0.0cm}{0.10cm}}
\pgfusepath{stroke}
}
}
%    \end{macrocode}
% \end{macro}
%
% \begin{macro}{cellHR}
%
%
%%%%%%%%%%%%%%%%%%%%%%%%%%%%%%%%%%%%%%%%%%%%%%%%%%%%%%%%%%%%%%%%%%%%%%%%%%%%%%%%%%%%%%%%%%%%%%%%%%%%%%%%%%%%%%%%%%%%%%%%%%%%%%%%%%%%%%%%%%%%%%%%%%%%%%%%%%%%%%%%%%%%%
%%
%% Shape: Cell (HR)
%%
%%%%%%%%%%%%%%%%%%%%%%%%%%%%%%%%%%%%%%%%%%%%%%%%%%%%%%%%%%%%%%%%%%%%%%%%%%%%%%%%%%%%%%%%%%%%%%%%%%%%%%%%%%%%%%%%%%%%%%%%%%%%%%%%%%%%%%%%%%%%%%%%%%%%%%%%%%%%%%%%%%%%%
%
%
%
% The |cellHR| shape is a cell with horizontal orientation and + terminal 
% rightward
%
%    \begin{macrocode}
\pgfdeclareshape{cellHR}{
\savedanchor\textpoint{%
\newdimen\ancx
\newdimen\ancy
\pgfextractx{\ancx}{\pgfpointorigin}%
\pgfextracty{\ancy}{\pgfpointorigin}%
\advance\ancy by -0.35cm%
\advance\ancy by -\ht\pgfnodeparttextbox%
\advance\ancx by -.5\wd\pgfnodeparttextbox%
\pgfpoint{\ancx}{\ancy}%
}
\savedanchor\westpoint{%
\newdimen\ancx%
\newdimen\ancy%
\pgfextractx{\ancx}{\pgfpointorigin}%
\pgfextracty{\ancy}{\pgfpointorigin}%
\advance\ancx by -0.1cm%
\pgfpoint{\ancx}{\ancy}%
}
\savedanchor\eastpoint{%
\newdimen\ancx%
\newdimen\ancy%
\pgfextractx{\ancx}{\pgfpointorigin}%
\pgfextracty{\ancy}{\pgfpointorigin}%
\advance\ancx by 0.1cm%
\pgfpoint{\ancx}{\ancy}%
}
\anchor{center}{\pgfpointorigin}
\anchor{west}{\westpoint}
\anchor{east}{\eastpoint}
\anchor{text}{\textpoint}
\anchorborder{
\newdimen\@tempdimxa%
\newdimen\@tempdimya%
\@tempdimxa=\pgf@x
\@tempdimya=\pgf@y
\pgfpointborderrectangle{\pgfpoint{\@tempdimxa}{\@tempdimya}}
{\pgfpoint{0.1cm}{0.3cm}}
}
\backgroundpath{
%%
\pgfpathmoveto{\pgfpoint{-0.05cm}{-0.2cm}}
\pgfpathlineto{\pgfpoint{-0.05cm}{0.2cm}}
%%
\pgfpathmoveto{\pgfpoint{0.05cm}{-0.3cm}}
\pgfpathlineto{\pgfpoint{0.05cm}{0.3cm}}
%%
\pgfpathmoveto{\pgfpoint{-0.05cm}{0.0cm}}
\pgfpathlineto{\pgfpoint{-0.10cm}{0.0cm}}
%%
\pgfpathmoveto{\pgfpoint{0.05cm}{0.0cm}}
\pgfpathlineto{\pgfpoint{0.10cm}{0.0cm}}
\pgfusepath{stroke}
}
}
%    \end{macrocode}
% \end{macro}
%
% \begin{macro}{cellHL}
%
%
%%%%%%%%%%%%%%%%%%%%%%%%%%%%%%%%%%%%%%%%%%%%%%%%%%%%%%%%%%%%%%%%%%%%%%%%%%%%%%%%%%%%%%%%%%%%%%%%%%%%%%%%%%%%%%%%%%%%%%%%%%%%%%%%%%%%%%%%%%%%%%%%%%%%%%%%%%%%%%%%%%%%%
%%
%% Shape: Cell (HL)
%%
%%%%%%%%%%%%%%%%%%%%%%%%%%%%%%%%%%%%%%%%%%%%%%%%%%%%%%%%%%%%%%%%%%%%%%%%%%%%%%%%%%%%%%%%%%%%%%%%%%%%%%%%%%%%%%%%%%%%%%%%%%%%%%%%%%%%%%%%%%%%%%%%%%%%%%%%%%%%%%%%%%%%%
%
%
%
% The |cellHL| shape is a cell with horizontal orientation and + terminal 
% leftward.
%
%    \begin{macrocode}
\pgfdeclareshape{cellHL}{
\savedanchor\textpoint{%
\newdimen\ancx
\newdimen\ancy
\pgfextractx{\ancx}{\pgfpointorigin}%
\pgfextracty{\ancy}{\pgfpointorigin}%
\advance\ancy by -0.35cm%
\advance\ancy by -\ht\pgfnodeparttextbox%
\advance\ancx by -.5\wd\pgfnodeparttextbox%
\pgfpoint{\ancx}{\ancy}%
}
\savedanchor\westpoint{%
\newdimen\ancx%
\newdimen\ancy%
\pgfextractx{\ancx}{\pgfpointorigin}%
\pgfextracty{\ancy}{\pgfpointorigin}%
\advance\ancx by -0.1cm%
\pgfpoint{\ancx}{\ancy}%
}
\savedanchor\eastpoint{%
\newdimen\ancx%
\newdimen\ancy%
\pgfextractx{\ancx}{\pgfpointorigin}%
\pgfextracty{\ancy}{\pgfpointorigin}%
\advance\ancx by 0.1cm%
\pgfpoint{\ancx}{\ancy}%
}
\anchor{center}{\pgfpointorigin}
\anchor{west}{\westpoint}
\anchor{east}{\eastpoint}
\anchor{text}{\textpoint}
\anchorborder{
\newdimen\@tempdimxa%
\newdimen\@tempdimya%
\@tempdimxa=\pgf@x
\@tempdimya=\pgf@y
\pgfpointborderrectangle{\pgfpoint{\@tempdimxa}{\@tempdimya}}
{\pgfpoint{0.10cm}{0.30cm}}
}
\backgroundpath{
%%
\pgfpathmoveto{\pgfpoint{-0.05cm}{-0.3cm}}
\pgfpathlineto{\pgfpoint{-0.05cm}{0.3cm}}
%%
\pgfpathmoveto{\pgfpoint{0.05cm}{-0.2cm}}
\pgfpathlineto{\pgfpoint{0.05cm}{0.2cm}}
%%
\pgfpathmoveto{\pgfpoint{-0.05cm}{0.0cm}}
\pgfpathlineto{\pgfpoint{-0.10cm}{0.0cm}}
%%
\pgfpathmoveto{\pgfpoint{0.05cm}{0.0cm}}
\pgfpathlineto{\pgfpoint{0.10cm}{0.0cm}}
\pgfusepath{stroke}
}
}
%    \end{macrocode}
% \end{macro}
%
% \begin{macro}{capacitorV}
%
%
%%%%%%%%%%%%%%%%%%%%%%%%%%%%%%%%%%%%%%%%%%%%%%%%%%%%%%%%%%%%%%%%%%%%%%%%%%%%%%%%%%%%%%%%%%%%%%%%%%%%%%%%%%%%%%%%%%%%%%%%%%%%%%%%%%%%%%%%%%%%%%%%%%%%%%%%%%%%%%%%%%%%%
%%
%% Shape: Capacitor (V)
%%
%%%%%%%%%%%%%%%%%%%%%%%%%%%%%%%%%%%%%%%%%%%%%%%%%%%%%%%%%%%%%%%%%%%%%%%%%%%%%%%%%%%%%%%%%%%%%%%%%%%%%%%%%%%%%%%%%%%%%%%%%%%%%%%%%%%%%%%%%%%%%%%%%%%%%%%%%%%%%%%%%%%%%
%
%
%
% The |capacitorV| shape is a capacitor with vertical orientation.
%
%    \begin{macrocode}
\pgfdeclareshape{capacitorV}{
\savedanchor\textpoint{%
\newdimen\ancx
\newdimen\ancy
\pgfextractx{\ancx}{\pgfpointorigin}%
\pgfextracty{\ancy}{\pgfpointorigin}%
\advance\ancx by 0.35cm%
\advance\ancy by -.5\ht\pgfnodeparttextbox%
\pgfpoint{\ancx}{\ancy}%
}
\savedanchor\northpoint{%
\newdimen\ancx%
\newdimen\ancy%
\pgfextractx{\ancx}{\pgfpointorigin}%
\pgfextracty{\ancy}{\pgfpointorigin}%
\advance\ancy by 0.1cm%
\pgfpoint{\ancx}{\ancy}%
}
\savedanchor\southpoint{%
\newdimen\ancx%
\newdimen\ancy%
\pgfextractx{\ancx}{\pgfpointorigin}%
\pgfextracty{\ancy}{\pgfpointorigin}%
\advance\ancy by -0.1cm%
\pgfpoint{\ancx}{\ancy}%
}
\anchor{center}{\pgfpointorigin}
\anchor{north}{\northpoint}
\anchor{south}{\southpoint}
\anchor{text}{\textpoint}
\anchorborder{
\newdimen\@tempdimxa%
\newdimen\@tempdimya%
\@tempdimxa=\pgf@x
\@tempdimya=\pgf@y
\pgfpointborderrectangle{\pgfpoint{\@tempdimxa}{\@tempdimya}}
{\pgfpoint{0.3cm}{0.10cm}}
}
\backgroundpath{
%% top plate
\pgfpathmoveto{\pgfpoint{-0.3cm}{-0.05cm}}
\pgfpathlineto{\pgfpoint{0.3cm}{-0.05cm}}
%% bottom plate
\pgfpathmoveto{\pgfpoint{-0.3cm}{0.05cm}}
\pgfpathlineto{\pgfpoint{0.3cm}{0.05cm}}
%% top lead
\pgfpathmoveto{\pgfpoint{0.0cm}{-0.05cm}}
\pgfpathlineto{\pgfpoint{0.0cm}{-0.10cm}}
%% bottom lead
\pgfpathmoveto{\pgfpoint{0.0cm}{0.05cm}}
\pgfpathlineto{\pgfpoint{0.0cm}{0.10cm}}
\pgfusepath{stroke}
}
}
%    \end{macrocode}
% \end{macro}
%
% \begin{macro}{capacitorH}
%
%
%%%%%%%%%%%%%%%%%%%%%%%%%%%%%%%%%%%%%%%%%%%%%%%%%%%%%%%%%%%%%%%%%%%%%%%%%%%%%%%%%%%%%%%%%%%%%%%%%%%%%%%%%%%%%%%%%%%%%%%%%%%%%%%%%%%%%%%%%%%%%%%%%%%%%%%%%%%%%%%%%%%%%
%%
%% Shape: Capacitor (H)
%%
%%%%%%%%%%%%%%%%%%%%%%%%%%%%%%%%%%%%%%%%%%%%%%%%%%%%%%%%%%%%%%%%%%%%%%%%%%%%%%%%%%%%%%%%%%%%%%%%%%%%%%%%%%%%%%%%%%%%%%%%%%%%%%%%%%%%%%%%%%%%%%%%%%%%%%%%%%%%%%%%%%%%%
%
%
%
% The |capacitorH| shape is a capacitor with horizontal orientation.
%
%    \begin{macrocode}
\pgfdeclareshape{capacitorH}{
\savedanchor\textpoint{%
\newdimen\ancx
\newdimen\ancy
\pgfextractx{\ancx}{\pgfpointorigin}%
\pgfextracty{\ancy}{\pgfpointorigin}%
\advance\ancy by -0.35cm%
\advance\ancy by -\ht\pgfnodeparttextbox%
\advance\ancx by -.5\wd\pgfnodeparttextbox%
\pgfpoint{\ancx}{\ancy}%
}
\savedanchor\westpoint{%
\newdimen\ancx%
\newdimen\ancy%
\pgfextractx{\ancx}{\pgfpointorigin}%
\pgfextracty{\ancy}{\pgfpointorigin}%
\advance\ancx by -0.1cm%
\pgfpoint{\ancx}{\ancy}%
}
\savedanchor\eastpoint{%
\newdimen\ancx%
\newdimen\ancy%
\pgfextractx{\ancx}{\pgfpointorigin}%
\pgfextracty{\ancy}{\pgfpointorigin}%
\advance\ancx by 0.1cm%
\pgfpoint{\ancx}{\ancy}%
}
\anchor{center}{\pgfpointorigin}
\anchor{west}{\westpoint}
\anchor{east}{\eastpoint}
\anchor{text}{\textpoint}
\anchorborder{
\newdimen\@tempdimxa%
\newdimen\@tempdimya%
\@tempdimxa=\pgf@x
\@tempdimya=\pgf@y
\pgfpointborderrectangle{\pgfpoint{\@tempdimxa}{\@tempdimya}}
{\pgfpoint{0.1cm}{0.30cm}}
}
\backgroundpath{
%% left plate
\pgfpathmoveto{\pgfpoint{-0.05cm}{-0.3cm}}
\pgfpathlineto{\pgfpoint{-0.05cm}{0.3cm}}
%% right plate
\pgfpathmoveto{\pgfpoint{0.05cm}{-0.3cm}}
\pgfpathlineto{\pgfpoint{0.05cm}{0.3cm}}
%% left lead
\pgfpathmoveto{\pgfpoint{-0.05cm}{0.0cm}}
\pgfpathlineto{\pgfpoint{-0.10cm}{0.0cm}}
%% right lead
\pgfpathmoveto{\pgfpoint{0.05cm}{0.0cm}}
\pgfpathlineto{\pgfpoint{0.10cm}{0.0cm}}
\pgfusepath{stroke}
}
}
%    \end{macrocode}
% \end{macro}
%
% \begin{macro}{resistorH}
%
%
%%%%%%%%%%%%%%%%%%%%%%%%%%%%%%%%%%%%%%%%%%%%%%%%%%%%%%%%%%%%%%%%%%%%%%%%%%%%%%%%%%%%%%%%%%%%%%%%%%%%%%%%%%%%%%%%%%%%%%%%%%%%%%%%%%%%%%%%%%%%%%%%%%%%%%%%%%%%%%%%%%%%%
%%
%% Shape: Resistor (H)
%%
%%%%%%%%%%%%%%%%%%%%%%%%%%%%%%%%%%%%%%%%%%%%%%%%%%%%%%%%%%%%%%%%%%%%%%%%%%%%%%%%%%%%%%%%%%%%%%%%%%%%%%%%%%%%%%%%%%%%%%%%%%%%%%%%%%%%%%%%%%%%%%%%%%%%%%%%%%%%%%%%%%%%%
%
%
%
% The |resistorH| shape is a resistor with horizontal orientation.
%
%    \begin{macrocode}
\pgfdeclareshape{resistorH}{
\savedanchor\textpoint{%
\newdimen\ancx
\newdimen\ancy
\pgfextractx{\ancx}{\pgfpointorigin}%
\pgfextracty{\ancy}{\pgfpointorigin}%
\advance\ancy by -0.10cm%
\advance\ancy by -\ht\pgfnodeparttextbox%
\advance\ancx by -.5\wd\pgfnodeparttextbox%
\pgfpoint{\ancx}{\ancy}%
}
\savedanchor\westpoint{%
\newdimen\ancx%
\newdimen\ancy%
\pgfextractx{\ancx}{\pgfpointorigin}%
\pgfextracty{\ancy}{\pgfpointorigin}%
\advance\ancx by -0.20cm%
\pgfpoint{\ancx}{\ancy}%
}
\savedanchor\eastpoint{%
\newdimen\ancx%
\newdimen\ancy%
\pgfextractx{\ancx}{\pgfpointorigin}%
\pgfextracty{\ancy}{\pgfpointorigin}%
\advance\ancx by 0.20cm%
\pgfpoint{\ancx}{\ancy}%
}
\anchor{center}{\pgfpointorigin}
\anchor{west}{\westpoint}
\anchor{east}{\eastpoint}
\anchor{text}{\textpoint}
\anchorborder{
\newdimen\@tempdimxa%
\newdimen\@tempdimya%
\@tempdimxa=\pgf@x
\@tempdimya=\pgf@y
\pgfpointborderrectangle{\pgfpoint{\@tempdimxa}{\@tempdimya}}
{\pgfpoint{0.2cm}{0.05cm}}
}
\backgroundpath{
%%
\pgfpathmoveto{\pgfpoint{-0.20cm}{0.0cm}}
\pgfpathlineto{\pgfpoint{-0.15cm}{0.0cm}}
\pgfpathlineto{\pgfpoint{-0.125cm}{0.05cm}}
\pgfpathlineto{\pgfpoint{-0.075cm}{-0.05cm}}
\pgfpathlineto{\pgfpoint{-0.025cm}{0.05cm}}
\pgfpathlineto{\pgfpoint{0.025cm}{-0.05cm}}
\pgfpathlineto{\pgfpoint{0.075cm}{0.05cm}}
\pgfpathlineto{\pgfpoint{0.125cm}{-0.05cm}}
\pgfpathlineto{\pgfpoint{0.15cm}{0.0cm}}
\pgfpathlineto{\pgfpoint{0.20cm}{0.0cm}}
%%
\pgfusepath{stroke}
}
}
%    \end{macrocode}
% \end{macro}
%
% \begin{macro}{resistorV}
%
%
%%%%%%%%%%%%%%%%%%%%%%%%%%%%%%%%%%%%%%%%%%%%%%%%%%%%%%%%%%%%%%%%%%%%%%%%%%%%%%%%%%%%%%%%%%%%%%%%%%%%%%%%%%%%%%%%%%%%%%%%%%%%%%%%%%%%%%%%%%%%%%%%%%%%%%%%%%%%%%%%%%%%%
%%
%% Shape: Resistor (V)
%%
%%%%%%%%%%%%%%%%%%%%%%%%%%%%%%%%%%%%%%%%%%%%%%%%%%%%%%%%%%%%%%%%%%%%%%%%%%%%%%%%%%%%%%%%%%%%%%%%%%%%%%%%%%%%%%%%%%%%%%%%%%%%%%%%%%%%%%%%%%%%%%%%%%%%%%%%%%%%%%%%%%%%%
%
%
%
% The |resistorV| shape is a resistor with vertical orientation.
%
%    \begin{macrocode}
\pgfdeclareshape{resistorV}{
\savedanchor\textpoint{%
\newdimen\ancx
\newdimen\ancy
\pgfextractx{\ancx}{\pgfpointorigin}%
\pgfextracty{\ancy}{\pgfpointorigin}%
\advance\ancx by 0.10cm%
\advance\ancy by -0.5\ht\pgfnodeparttextbox%
\pgfpoint{\ancx}{\ancy}%
}
\savedanchor\northpoint{%
\newdimen\ancx%
\newdimen\ancy%
\pgfextractx{\ancx}{\pgfpointorigin}%
\pgfextracty{\ancy}{\pgfpointorigin}%
\advance\ancy by 0.20cm%
\pgfpoint{\ancx}{\ancy}%
}
\savedanchor\southpoint{%
\newdimen\ancx%
\newdimen\ancy%
\pgfextractx{\ancx}{\pgfpointorigin}%
\pgfextracty{\ancy}{\pgfpointorigin}%
\advance\ancy by -0.20cm%
\pgfpoint{\ancx}{\ancy}%
}
\anchor{center}{\pgfpointorigin}
\anchor{north}{\northpoint}
\anchor{south}{\southpoint}
\anchor{text}{\textpoint}
\anchorborder{
\newdimen\@tempdimxa%
\newdimen\@tempdimya%
\@tempdimxa=\pgf@x
\@tempdimya=\pgf@y
\pgfpointborderrectangle{\pgfpoint{\@tempdimxa}{\@tempdimya}}
{\pgfpoint{0.05cm}{0.2cm}}
}
\backgroundpath{
%%
\pgfpathmoveto{\pgfpoint{0.0cm}{-0.20cm}}
\pgfpathlineto{\pgfpoint{0.0cm}{-0.15cm}}
\pgfpathlineto{\pgfpoint{0.05cm}{-0.125cm}}
\pgfpathlineto{\pgfpoint{-0.05cm}{-0.075cm}}
\pgfpathlineto{\pgfpoint{0.05cm}{-0.025cm}}
\pgfpathlineto{\pgfpoint{-0.05cm}{0.025cm}}
\pgfpathlineto{\pgfpoint{0.05cm}{0.075cm}}
\pgfpathlineto{\pgfpoint{-0.05cm}{0.125cm}}
\pgfpathlineto{\pgfpoint{0.0cm}{0.15cm}}
\pgfpathlineto{\pgfpoint{0.0cm}{0.20cm}}
%
\pgfusepath{stroke}
}
}
%    \end{macrocode}
% \end{macro}
%
% \begin{macro}{inductorH}
%
%
%%%%%%%%%%%%%%%%%%%%%%%%%%%%%%%%%%%%%%%%%%%%%%%%%%%%%%%%%%%%%%%%%%%%%%%%%%%%%%%%%%%%%%%%%%%%%%%%%%%%%%%%%%%%%%%%%%%%%%%%%%%%%%%%%%%%%%%%%%%%%%%%%%%%%%%%%%%%%%%%%%%%%
%%
%% Shape: Inductor (H)
%%
%%%%%%%%%%%%%%%%%%%%%%%%%%%%%%%%%%%%%%%%%%%%%%%%%%%%%%%%%%%%%%%%%%%%%%%%%%%%%%%%%%%%%%%%%%%%%%%%%%%%%%%%%%%%%%%%%%%%%%%%%%%%%%%%%%%%%%%%%%%%%%%%%%%%%%%%%%%%%%%%%%%%%
%
%
%
% The |inductorH| shape is an inductor with horizontal orientation.
%
%    \begin{macrocode}
\pgfdeclareshape{inductorH}{
\savedanchor\textpoint{%
\newdimen\ancx
\newdimen\ancy
\pgfextractx{\ancx}{\pgfpointorigin}%
\pgfextracty{\ancy}{\pgfpointorigin}%
\advance\ancy by -0.05cm%
\advance\ancy by -\ht\pgfnodeparttextbox%
\advance\ancx by -.5\wd\pgfnodeparttextbox%
\pgfpoint{\ancx}{\ancy}%
}
\savedanchor\westpoint{%
\newdimen\ancx%
\newdimen\ancy%
\pgfextractx{\ancx}{\pgfpointorigin}%
\pgfextracty{\ancy}{\pgfpointorigin}%
\advance\ancx by -0.35cm%
\pgfpoint{\ancx}{\ancy}%
}
\savedanchor\eastpoint{%
\newdimen\ancx%
\newdimen\ancy%
\pgfextractx{\ancx}{\pgfpointorigin}%
\pgfextracty{\ancy}{\pgfpointorigin}%
\advance\ancx by 0.35cm%
\pgfpoint{\ancx}{\ancy}%
}
\anchor{center}{\pgfpointorigin}
\anchor{west}{\westpoint}
\anchor{east}{\eastpoint}
\anchor{text}{\textpoint}
\anchorborder{
\newdimen\@tempdimxa%
\newdimen\@tempdimya%
\@tempdimxa=\pgf@x
\@tempdimya=\pgf@y
\pgfpointborderrectangle{\pgfpoint{\@tempdimxa}{\@tempdimya}}
{\pgfpoint{0.35cm}{0.1cm}}
}
\backgroundpath{
%%
\pgfpathmoveto{\pgfpoint{0.35cm}{0.0cm}}
\pgfpathlineto{\pgfpoint{0.3cm}{0.0cm}}
\pgfpatharcto{0.1cm}{0.1cm}{0}{0}{1}{\pgfpoint{0.1cm}{0.0cm}}
\pgfpatharcto{0.1cm}{0.1cm}{0}{0}{1}{\pgfpoint{-0.1cm}{0.0cm}}
\pgfpatharcto{0.1cm}{0.1cm}{0}{0}{1}{\pgfpoint{-0.3cm}{0.0cm}}
\pgfpathlineto{\pgfpoint{-0.35cm}{0.0cm}}
%%
\pgfusepath{stroke}
}
}
%    \end{macrocode}
% \end{macro}
%
% \begin{macro}{inductorV}
%
%
%%%%%%%%%%%%%%%%%%%%%%%%%%%%%%%%%%%%%%%%%%%%%%%%%%%%%%%%%%%%%%%%%%%%%%%%%%%%%%%%%%%%%%%%%%%%%%%%%%%%%%%%%%%%%%%%%%%%%%%%%%%%%%%%%%%%%%%%%%%%%%%%%%%%%%%%%%%%%%%%%%%%%
%%
%% Shape: Inductor (V)
%%
%%%%%%%%%%%%%%%%%%%%%%%%%%%%%%%%%%%%%%%%%%%%%%%%%%%%%%%%%%%%%%%%%%%%%%%%%%%%%%%%%%%%%%%%%%%%%%%%%%%%%%%%%%%%%%%%%%%%%%%%%%%%%%%%%%%%%%%%%%%%%%%%%%%%%%%%%%%%%%%%%%%%%
%
%
%
% The |inductorV| shape is an inductor with vertical orientation.
%
%    \begin{macrocode}
\pgfdeclareshape{inductorV}{
\savedanchor\textpoint{%
\newdimen\ancx
\newdimen\ancy
\pgfextractx{\ancx}{\pgfpointorigin}%
\pgfextracty{\ancy}{\pgfpointorigin}%
\advance\ancx by 0.15cm%
\advance\ancy by -0.5\ht\pgfnodeparttextbox%
\pgfpoint{\ancx}{\ancy}%
}
\savedanchor\northpoint{%
\newdimen\ancx%
\newdimen\ancy%
\pgfextractx{\ancx}{\pgfpointorigin}%
\pgfextracty{\ancy}{\pgfpointorigin}%
\advance\ancy by 0.35cm%
\pgfpoint{\ancx}{\ancy}%
}
\savedanchor\southpoint{%
\newdimen\ancx%
\newdimen\ancy%
\pgfextractx{\ancx}{\pgfpointorigin}%
\pgfextracty{\ancy}{\pgfpointorigin}%
\advance\ancy by -0.35cm%
\pgfpoint{\ancx}{\ancy}%
}
\anchor{center}{\pgfpointorigin}
\anchor{north}{\northpoint}
\anchor{south}{\southpoint}
\anchor{text}{\textpoint}
\anchorborder{
\newdimen\@tempdimxa%
\newdimen\@tempdimya%
\@tempdimxa=\pgf@x
\@tempdimya=\pgf@y
\pgfpointborderrectangle{\pgfpoint{\@tempdimxa}{\@tempdimya}}
{\pgfpoint{0.1cm}{0.35cm}}
}
\backgroundpath{
%%
\pgfpathmoveto{\pgfpoint{0.0cm}{0.35cm}}
\pgfpathlineto{\pgfpoint{0.0cm}{0.3cm}}
\pgfpatharcto{0.1cm}{0.1cm}{0}{0}{0}{\pgfpoint{0.0cm}{0.1cm}}
\pgfpatharcto{0.1cm}{0.1cm}{0}{0}{1}{\pgfpoint{0.0cm}{-0.1cm}}
\pgfpatharcto{0.1cm}{0.1cm}{0}{0}{0}{\pgfpoint{0.0cm}{-0.3cm}}
\pgfpathlineto{\pgfpoint{0.0cm}{-0.35cm}}
%
\pgfusepath{stroke}
}
}
%    \end{macrocode}
% \end{macro}
%
% \begin{macro}{transformerH}
%
%
%%%%%%%%%%%%%%%%%%%%%%%%%%%%%%%%%%%%%%%%%%%%%%%%%%%%%%%%%%%%%%%%%%%%%%%%%%%%%%%%%%%%%%%%%%%%%%%%%%%%%%%%%%%%%%%%%%%%%%%%%%%%%%%%%%%%%%%%%%%%%%%%%%%%%%%%%%%%%%%%%%%%%
%%
%% Shape: Transformer (H)
%%
%%%%%%%%%%%%%%%%%%%%%%%%%%%%%%%%%%%%%%%%%%%%%%%%%%%%%%%%%%%%%%%%%%%%%%%%%%%%%%%%%%%%%%%%%%%%%%%%%%%%%%%%%%%%%%%%%%%%%%%%%%%%%%%%%%%%%%%%%%%%%%%%%%%%%%%%%%%%%%%%%%%%%
%
%
%
% The |transformerH| shape is an transformer with horizontal orientation (leads 
% on the left and right).
%
%    \begin{macrocode}
\pgfdeclareshape{transformerH}{
\savedanchor\textpoint{%
\newdimen\ancx
\newdimen\ancy
\pgfextractx{\ancx}{\pgfpointorigin}%
\pgfextracty{\ancy}{\pgfpointorigin}%
\advance\ancx by -0.5\wd\pgfnodeparttextbox%
\advance\ancy by -0.45cm
\advance\ancy by -\ht\pgfnodeparttextbox%
\pgfpoint{\ancx}{\ancy}%
}
\savedanchor\northeastpoint{%
\newdimen\ancx%
\newdimen\ancy%
\pgfextractx{\ancx}{\pgfpointorigin}%
\pgfextracty{\ancy}{\pgfpointorigin}%
\advance\ancy by 0.40cm%
\advance\ancx by 0.35cm%
\pgfpoint{\ancx}{\ancy}%
}
\savedanchor\southeastpoint{%
\newdimen\ancx%
\newdimen\ancy%
\pgfextractx{\ancx}{\pgfpointorigin}%
\pgfextracty{\ancy}{\pgfpointorigin}%
\advance\ancy by -0.40cm%
\advance\ancx by 0.35cm%
\pgfpoint{\ancx}{\ancy}%
}
\savedanchor\northwestpoint{%
\newdimen\ancx%
\newdimen\ancy%
\pgfextractx{\ancx}{\pgfpointorigin}%
\pgfextracty{\ancy}{\pgfpointorigin}%
\advance\ancy by 0.40cm%
\advance\ancx by -0.35cm%
\pgfpoint{\ancx}{\ancy}%
}
\savedanchor\southwestpoint{%
\newdimen\ancx%
\newdimen\ancy%
\pgfextractx{\ancx}{\pgfpointorigin}%
\pgfextracty{\ancy}{\pgfpointorigin}%
\advance\ancy by -0.40cm%
\advance\ancx by -0.35cm%
\pgfpoint{\ancx}{\ancy}%
}
\anchor{center}{\pgfpointorigin}
\anchor{northeast}{\northeastpoint}
\anchor{southeast}{\southeastpoint}
\anchor{northwest}{\northwestpoint}
\anchor{southwest}{\southwestpoint}
\anchor{text}{\textpoint}
\anchorborder{
\newdimen\@tempdimxa%
\newdimen\@tempdimya%
\@tempdimxa=\pgf@x
\@tempdimya=\pgf@y
\pgfpointborderrectangle{\pgfpoint{\@tempdimxa}{\@tempdimya}}
{\pgfpoint{0.35cm}{0.40cm}}
}
\backgroundpath{
%% right inductor
\pgfpathmoveto{\pgfpoint{0.35cm}{0.40cm}}
\pgfpathlineto{\pgfpoint{0.25cm}{0.40cm}}
\pgfpathlineto{\pgfpoint{0.25cm}{0.3cm}}
\pgfpatharcto{0.1cm}{0.1cm}{0}{0}{1}{\pgfpoint{0.15cm}{0.2cm}}
\pgfpatharcto{0.1cm}{0.1cm}{0}{0}{1}{\pgfpoint{0.25cm}{0.1cm}}
\pgfpatharcto{0.1cm}{0.1cm}{0}{0}{1}{\pgfpoint{0.15cm}{0.0cm}}
\pgfpatharcto{0.1cm}{0.1cm}{0}{0}{1}{\pgfpoint{0.25cm}{-0.1cm}}
\pgfpatharcto{0.1cm}{0.1cm}{0}{0}{1}{\pgfpoint{0.15cm}{-0.2cm}}
\pgfpatharcto{0.1cm}{0.1cm}{0}{0}{1}{\pgfpoint{0.25cm}{-0.3cm}}
\pgfpathlineto{\pgfpoint{0.25cm}{-0.40cm}}
\pgfpathlineto{\pgfpoint{0.35cm}{-0.40cm}}
%% left inductor
\pgfpathmoveto{\pgfpoint{-0.35cm}{0.40cm}}
\pgfpathlineto{\pgfpoint{-0.25cm}{0.40cm}}
\pgfpathlineto{\pgfpoint{-0.25cm}{0.3cm}}
\pgfpatharcto{0.1cm}{0.1cm}{0}{0}{0}{\pgfpoint{-0.15cm}{0.2cm}}
\pgfpatharcto{0.1cm}{0.1cm}{0}{0}{0}{\pgfpoint{-0.25cm}{0.1cm}}
\pgfpatharcto{0.1cm}{0.1cm}{0}{0}{0}{\pgfpoint{-0.15cm}{0.0cm}}
\pgfpatharcto{0.1cm}{0.1cm}{0}{0}{0}{\pgfpoint{-0.25cm}{-0.1cm}}
\pgfpatharcto{0.1cm}{0.1cm}{0}{0}{0}{\pgfpoint{-0.15cm}{-0.2cm}}
\pgfpatharcto{0.1cm}{0.1cm}{0}{0}{0}{\pgfpoint{-0.25cm}{-0.3cm}}
\pgfpathlineto{\pgfpoint{-0.25cm}{-0.40cm}}
\pgfpathlineto{\pgfpoint{-0.35cm}{-0.40cm}}
%% left plate
\pgfpathmoveto{\pgfpoint{-0.05cm}{0.2cm}}
\pgfpathlineto{\pgfpoint{-0.05cm}{-0.2cm}}
%% right plate
\pgfpathmoveto{\pgfpoint{0.05cm}{0.2cm}}
\pgfpathlineto{\pgfpoint{0.05cm}{-0.2cm}}
\pgfusepath{stroke}
}
}
%    \end{macrocode}
% \end{macro}
%
% \begin{macro}{transformerV}
%
%
%%%%%%%%%%%%%%%%%%%%%%%%%%%%%%%%%%%%%%%%%%%%%%%%%%%%%%%%%%%%%%%%%%%%%%%%%%%%%%%%%%%%%%%%%%%%%%%%%%%%%%%%%%%%%%%%%%%%%%%%%%%%%%%%%%%%%%%%%%%%%%%%%%%%%%%%%%%%%%%%%%%%%
%%
%% Shape: Transformer (V)
%%
%%%%%%%%%%%%%%%%%%%%%%%%%%%%%%%%%%%%%%%%%%%%%%%%%%%%%%%%%%%%%%%%%%%%%%%%%%%%%%%%%%%%%%%%%%%%%%%%%%%%%%%%%%%%%%%%%%%%%%%%%%%%%%%%%%%%%%%%%%%%%%%%%%%%%%%%%%%%%%%%%%%%%
%
%
%
% The |transformerV| shape is a transformer with vertical orientation (leads on 
% top and bottom).
%
%    \begin{macrocode}
\pgfdeclareshape{transformerV}{
\savedanchor\textpoint{%
\newdimen\ancx
\newdimen\ancy
\pgfextractx{\ancx}{\pgfpointorigin}%
\pgfextracty{\ancy}{\pgfpointorigin}%
\advance\ancx by 0.40cm
\advance\ancy by -0.5\ht\pgfnodeparttextbox%
\pgfpoint{\ancx}{\ancy}%
}
\savedanchor\northeastpoint{%
\newdimen\ancx%
\newdimen\ancy%
\pgfextractx{\ancx}{\pgfpointorigin}%
\pgfextracty{\ancy}{\pgfpointorigin}%
\advance\ancy by 0.35cm%
\advance\ancx by 0.40cm%
\pgfpoint{\ancx}{\ancy}%
}
\savedanchor\southeastpoint{%
\newdimen\ancx%
\newdimen\ancy%
\pgfextractx{\ancx}{\pgfpointorigin}%
\pgfextracty{\ancy}{\pgfpointorigin}%
\advance\ancy by -0.35cm%
\advance\ancx by 0.40cm%
\pgfpoint{\ancx}{\ancy}%
}
\savedanchor\northwestpoint{%
\newdimen\ancx%
\newdimen\ancy%
\pgfextractx{\ancx}{\pgfpointorigin}%
\pgfextracty{\ancy}{\pgfpointorigin}%
\advance\ancy by 0.35cm%
\advance\ancx by -0.40cm%
\pgfpoint{\ancx}{\ancy}%
}
\savedanchor\southwestpoint{%
\newdimen\ancx%
\newdimen\ancy%
\pgfextractx{\ancx}{\pgfpointorigin}%
\pgfextracty{\ancy}{\pgfpointorigin}%
\advance\ancy by -0.35cm%
\advance\ancx by -0.40cm%
\pgfpoint{\ancx}{\ancy}%
}
\anchor{center}{\pgfpointorigin}
\anchor{northeast}{\northeastpoint}
\anchor{southeast}{\southeastpoint}
\anchor{northwest}{\northwestpoint}
\anchor{southwest}{\southwestpoint}
\anchor{text}{\textpoint}
\anchorborder{
\newdimen\@tempdimxa%
\newdimen\@tempdimya%
\@tempdimxa=\pgf@x
\@tempdimya=\pgf@y
\pgfpointborderrectangle{\pgfpoint{\@tempdimxa}{\@tempdimya}}
{\pgfpoint{0.40cm}{0.35cm}}
}
\backgroundpath{
%% top inductor
\pgfpathmoveto{\pgfpoint{0.40cm}{0.35cm}}
\pgfpathlineto{\pgfpoint{0.40cm}{0.25cm}}
\pgfpathlineto{\pgfpoint{0.3cm}{0.25cm}}
\pgfpatharcto{0.1cm}{0.1cm}{0}{0}{0}{\pgfpoint{0.2cm}{0.15cm}}
\pgfpatharcto{0.1cm}{0.1cm}{0}{0}{0}{\pgfpoint{0.1cm}{0.25cm}}
\pgfpatharcto{0.1cm}{0.1cm}{0}{0}{0}{\pgfpoint{0.0cm}{0.15cm}}
\pgfpatharcto{0.1cm}{0.1cm}{0}{0}{0}{\pgfpoint{-0.1cm}{0.25cm}}
\pgfpatharcto{0.1cm}{0.1cm}{0}{0}{0}{\pgfpoint{-0.2cm}{0.15cm}}
\pgfpatharcto{0.1cm}{0.1cm}{0}{0}{0}{\pgfpoint{-0.3cm}{0.25cm}}
\pgfpathlineto{\pgfpoint{-0.40cm}{0.25cm}}
\pgfpathlineto{\pgfpoint{-0.40cm}{0.35cm}}
%% bottom inductor
\pgfpathmoveto{\pgfpoint{0.40cm}{-0.35cm}}
\pgfpathlineto{\pgfpoint{0.40cm}{-0.25cm}}
\pgfpathlineto{\pgfpoint{0.3cm}{-0.25cm}}
\pgfpatharcto{0.1cm}{0.1cm}{0}{0}{1}{\pgfpoint{0.2cm}{-0.15cm}}
\pgfpatharcto{0.1cm}{0.1cm}{0}{0}{1}{\pgfpoint{0.1cm}{-0.25cm}}
\pgfpatharcto{0.1cm}{0.1cm}{0}{0}{1}{\pgfpoint{0.0cm}{-0.15cm}}
\pgfpatharcto{0.1cm}{0.1cm}{0}{0}{1}{\pgfpoint{-0.1cm}{-0.25cm}}
\pgfpatharcto{0.1cm}{0.1cm}{0}{0}{1}{\pgfpoint{-0.2cm}{-0.15cm}}
\pgfpatharcto{0.1cm}{0.1cm}{0}{0}{1}{\pgfpoint{-0.3cm}{-0.25cm}}
\pgfpathlineto{\pgfpoint{-0.40cm}{-0.25cm}}
\pgfpathlineto{\pgfpoint{-0.40cm}{-0.35cm}}
%% bottom plate
\pgfpathmoveto{\pgfpoint{0.2cm}{-0.05cm}}
\pgfpathlineto{\pgfpoint{-0.2cm}{-0.05cm}}
%% top plate
\pgfpathmoveto{\pgfpoint{0.2cm}{0.05cm}}
\pgfpathlineto{\pgfpoint{-0.2cm}{0.05cm}}
\pgfusepath{stroke}
}
}
%    \end{macrocode}
% \end{macro}
%
% \begin{macro}{ACsource}
%
%
%%%%%%%%%%%%%%%%%%%%%%%%%%%%%%%%%%%%%%%%%%%%%%%%%%%%%%%%%%%%%%%%%%%%%%%%%%%%%%%%%%%%%%%%%%%%%%%%%%%%%%%%%%%%%%%%%%%%%%%%%%%%%%%%%%%%%%%%%%%%%%%%%%%%%%%%%%%%%%%%%%%%%
%%
%% Shape: AC Source
%%
%%%%%%%%%%%%%%%%%%%%%%%%%%%%%%%%%%%%%%%%%%%%%%%%%%%%%%%%%%%%%%%%%%%%%%%%%%%%%%%%%%%%%%%%%%%%%%%%%%%%%%%%%%%%%%%%%%%%%%%%%%%%%%%%%%%%%%%%%%%%%%%%%%%%%%%%%%%%%%%%%%%%%
%
%
%
% The |ACsource| shape is an alternate current source.
%
%    \begin{macrocode}
\pgfdeclareshape{ACsource}{
\savedanchor\textpoint{%
\newdimen\ancx
\newdimen\ancy
\pgfextractx{\ancx}{\pgfpointorigin}%
\pgfextracty{\ancy}{\pgfpointorigin}%
\advance\ancx by -0.5\wd\pgfnodeparttextbox%
\advance\ancy by -0.25cm
\advance\ancy by -\ht\pgfnodeparttextbox%
\pgfpoint{\ancx}{\ancy}%
}
%%
\savedanchor\northpoint{%
\newdimen\ancx%
\newdimen\ancy%
\pgfextractx{\ancx}{\pgfpointorigin}%
\pgfextracty{\ancy}{\pgfpointorigin}%
\advance\ancy by 0.25cm%
\pgfpoint{\ancx}{\ancy}%
}
\savedanchor\southpoint{%
\newdimen\ancx%
\newdimen\ancy%
\pgfextractx{\ancx}{\pgfpointorigin}%
\pgfextracty{\ancy}{\pgfpointorigin}%
\advance\ancy by -0.25cm%
\pgfpoint{\ancx}{\ancy}%
}
\savedanchor\eastpoint{%
\newdimen\ancx%
\newdimen\ancy%
\pgfextractx{\ancx}{\pgfpointorigin}%
\pgfextracty{\ancy}{\pgfpointorigin}%
\advance\ancx by 0.25cm%
\pgfpoint{\ancx}{\ancy}%
}
\savedanchor\westpoint{%
\newdimen\ancx%
\newdimen\ancy%
\pgfextractx{\ancx}{\pgfpointorigin}%
\pgfextracty{\ancy}{\pgfpointorigin}%
\advance\ancx by -0.25cm%
\pgfpoint{\ancx}{\ancy}%
}
\anchor{center}{\pgfpointorigin}
\anchor{text}{\textpoint}
\anchor{east}{\eastpoint}
\anchor{west}{\westpoint}
\anchor{north}{\northpoint}
\anchor{south}{\southpoint}
\anchorborder{%
\newdimen\@tempdimxa
\newdimen\@tempdimya
\@tempdimxa=\pgf@x
\@tempdimya=\pgf@y
\pgfpointborderellipse{\pgfpoint{\@tempdimxa}{\@tempdimya}}
{\pgfpoint{0.25cm}{0.25cm}}
}
%%
\backgroundpath{
\pgfpathcircle{\pgfpointorigin}{0.25cm}%
%% draw the AC symbol
\pgfpathmoveto{\pgfpoint{-0.2cm}{0.0cm}}%
\pgfpathsine{\pgfpoint{0.1cm}{0.1cm}}%
\pgfpathcosine{\pgfpoint{0.1cm}{-0.1cm}}%
\pgfpathsine{\pgfpoint{0.1cm}{-0.1cm}}%
\pgfpathcosine{\pgfpoint{0.1cm}{0.1cm}}%
\pgfusepath{stroke}
}
}
%    \end{macrocode}
% \end{macro}
%
% \begin{macro}{DCsource}
%
%
%%%%%%%%%%%%%%%%%%%%%%%%%%%%%%%%%%%%%%%%%%%%%%%%%%%%%%%%%%%%%%%%%%%%%%%%%%%%%%%%%%%%%%%%%%%%%%%%%%%%%%%%%%%%%%%%%%%%%%%%%%%%%%%%%%%%%%%%%%%%%%%%%%%%%%%%%%%%%%%%%%%%%
%%
%% Shape: DC Source
%%
%%%%%%%%%%%%%%%%%%%%%%%%%%%%%%%%%%%%%%%%%%%%%%%%%%%%%%%%%%%%%%%%%%%%%%%%%%%%%%%%%%%%%%%%%%%%%%%%%%%%%%%%%%%%%%%%%%%%%%%%%%%%%%%%%%%%%%%%%%%%%%%%%%%%%%%%%%%%%%%%%%%%%
%
%
%
% The |DCsource| shape is a direct current source.
%
%    \begin{macrocode}
\pgfdeclareshape{DCsource}{
\savedanchor\textpoint{%
\newdimen\ancx
\newdimen\ancy
\pgfextractx{\ancx}{\pgfpointorigin}%
\pgfextracty{\ancy}{\pgfpointorigin}%
\advance\ancx by -0.5\wd\pgfnodeparttextbox%
\advance\ancy by -0.25cm
\advance\ancy by -\ht\pgfnodeparttextbox%
\pgfpoint{\ancx}{\ancy}%
}
\savedanchor\eastpoint{%
\newdimen\ancx%
\newdimen\ancy%
\pgfextractx{\ancx}{\pgfpointorigin}%
\pgfextracty{\ancy}{\pgfpointorigin}%
\advance\ancx by 0.25cm%
\pgfpoint{\ancx}{\ancy}%
}
\savedanchor\westpoint{%
\newdimen\ancx%
\newdimen\ancy%
\pgfextractx{\ancx}{\pgfpointorigin}%
\pgfextracty{\ancy}{\pgfpointorigin}%
\advance\ancx by -0.25cm%
\pgfpoint{\ancx}{\ancy}%
}
\savedanchor\northpoint{%
\newdimen\ancx%
\newdimen\ancy%
\pgfextractx{\ancx}{\pgfpointorigin}%
\pgfextracty{\ancy}{\pgfpointorigin}%
\advance\ancy by 0.30cm%
\pgfpoint{\ancx}{\ancy}%
}
\savedanchor\southpoint{%
\newdimen\ancx%
\newdimen\ancy%
\pgfextractx{\ancx}{\pgfpointorigin}%
\pgfextracty{\ancy}{\pgfpointorigin}%
\advance\ancy by -0.25cm%
\pgfpoint{\ancx}{\ancy}%
}
\anchor{center}{\pgfpointorigin}
\anchor{text}{\textpoint}
\anchor{east}{\eastpoint}
\anchor{west}{\westpoint}
\anchor{north}{\northpoint}
\anchor{south}{\southpoint}
\anchorborder{%
\newdimen\@tempdimxa
\newdimen\@tempdimya
\@tempdimxa=\pgf@x
\@tempdimya=\pgf@y
\pgfpointborderellipse{\pgfpoint{\@tempdimxa}{\@tempdimya}}
{\pgfpoint{0.25cm}{0.25cm}}
}
%%
\backgroundpath{
\pgfpathcircle{\pgfpointorigin}{0.25cm}%
%% long line (part of DC symbol)
\pgfpathmoveto{\pgfpoint{-0.1cm}{0.05cm}}%
\pgfpathlineto{\pgfpoint{0.1cm}{0.05cm}}%
%% left short segment
\pgfpathmoveto{\pgfpoint{-0.1cm}{-0.05cm}}%
\pgfpathlineto{\pgfpoint{-0.06cm}{-0.05cm}}%
%% center short segment
\pgfpathmoveto{\pgfpoint{-0.02cm}{-0.05cm}}%
\pgfpathlineto{\pgfpoint{0.02cm}{-0.05cm}}%
%% right short segment
\pgfpathmoveto{\pgfpoint{0.06cm}{-0.05cm}}%
\pgfpathlineto{\pgfpoint{0.10cm}{-0.05cm}}%
\pgfusepath{stroke}
}
}
%    \end{macrocode}
% \end{macro}
%
% \begin{macro}{ammeter}
%
%
%%%%%%%%%%%%%%%%%%%%%%%%%%%%%%%%%%%%%%%%%%%%%%%%%%%%%%%%%%%%%%%%%%%%%%%%%%%%%%%%%%%%%%%%%%%%%%%%%%%%%%%%%%%%%%%%%%%%%%%%%%%%%%%%%%%%%%%%%%%%%%%%%%%%%%%%%%%%%%%%%%%%%
%%
%% Shape: Ammeter
%%
%%%%%%%%%%%%%%%%%%%%%%%%%%%%%%%%%%%%%%%%%%%%%%%%%%%%%%%%%%%%%%%%%%%%%%%%%%%%%%%%%%%%%%%%%%%%%%%%%%%%%%%%%%%%%%%%%%%%%%%%%%%%%%%%%%%%%%%%%%%%%%%%%%%%%%%%%%%%%%%%%%%%%
%
%
%
% The |ammeter| shape is an ammeter.
%
%    \begin{macrocode}
\pgfdeclareshape{ammeter}{
\savedanchor\textpoint{%
\newdimen\ancx
\newdimen\ancy
\pgfextractx{\ancx}{\pgfpointorigin}%
\pgfextracty{\ancy}{\pgfpointorigin}%
\advance\ancx by -0.5\wd\pgfnodeparttextbox%
\advance\ancy by -0.25cm
\advance\ancy by -\ht\pgfnodeparttextbox%
\pgfpoint{\ancx}{\ancy}%
}
%%
\savedanchor\northpoint{%
\newdimen\ancx%
\newdimen\ancy%
\pgfextractx{\ancx}{\pgfpointorigin}%
\pgfextracty{\ancy}{\pgfpointorigin}%
\advance\ancy by 0.25cm%
\pgfpoint{\ancx}{\ancy}%
}
\savedanchor\southpoint{%
\newdimen\ancx%
\newdimen\ancy%
\pgfextractx{\ancx}{\pgfpointorigin}%
\pgfextracty{\ancy}{\pgfpointorigin}%
\advance\ancy by -0.25cm%
\pgfpoint{\ancx}{\ancy}%
}
\savedanchor\eastpoint{%
\newdimen\ancx%
\newdimen\ancy%
\pgfextractx{\ancx}{\pgfpointorigin}%
\pgfextracty{\ancy}{\pgfpointorigin}%
\advance\ancx by 0.25cm%
\pgfpoint{\ancx}{\ancy}%
}
\savedanchor\westpoint{%
\newdimen\ancx%
\newdimen\ancy%
\pgfextractx{\ancx}{\pgfpointorigin}%
\pgfextracty{\ancy}{\pgfpointorigin}%
\advance\ancx by -0.25cm%
\pgfpoint{\ancx}{\ancy}%
}
\anchor{center}{\pgfpointorigin}
\anchor{text}{\textpoint}
\anchor{east}{\eastpoint}
\anchor{west}{\westpoint}
\anchor{north}{\northpoint}
\anchor{south}{\southpoint}
\anchorborder{%
\newdimen\@tempdimxa
\newdimen\@tempdimya
\@tempdimxa=\pgf@x
\@tempdimya=\pgf@y
\pgfpointborderellipse{\pgfpoint{\@tempdimxa}{\@tempdimya}}
{\pgfpoint{0.25cm}{0.25cm}}
}
%%
\backgroundpath{
\pgfpathcircle{\pgfpointorigin}{0.25cm}%
%    \end{macrocode}
% get the current color to make sure that the text in the center  matches the 
% desired stroke color
%    \begin{macrocode}
\colorlet{saved}{.}
\pgftext{\color{saved} A}%
\pgfusepath{stroke}
}
}
%    \end{macrocode}
% \end{macro}
%
% \begin{macro}{voltmeter}
%
%
%%%%%%%%%%%%%%%%%%%%%%%%%%%%%%%%%%%%%%%%%%%%%%%%%%%%%%%%%%%%%%%%%%%%%%%%%%%%%%%%%%%%%%%%%%%%%%%%%%%%%%%%%%%%%%%%%%%%%%%%%%%%%%%%%%%%%%%%%%%%%%%%%%%%%%%%%%%%%%%%%%%%%
%%
%% Shape: Voltmeter
%%
%%%%%%%%%%%%%%%%%%%%%%%%%%%%%%%%%%%%%%%%%%%%%%%%%%%%%%%%%%%%%%%%%%%%%%%%%%%%%%%%%%%%%%%%%%%%%%%%%%%%%%%%%%%%%%%%%%%%%%%%%%%%%%%%%%%%%%%%%%%%%%%%%%%%%%%%%%%%%%%%%%%%%
%
%
%
% The |voltmeter| shape is a voltmeter.
%
%    \begin{macrocode}
\pgfdeclareshape{voltmeter}{
\savedanchor\textpoint{%
\newdimen\ancx
\newdimen\ancy
\pgfextractx{\ancx}{\pgfpointorigin}%
\pgfextracty{\ancy}{\pgfpointorigin}%
\advance\ancx by -0.5\wd\pgfnodeparttextbox%
\advance\ancy by -0.25cm
\advance\ancy by -\ht\pgfnodeparttextbox%
\pgfpoint{\ancx}{\ancy}%
}
%
\savedanchor\northpoint{%
\newdimen\ancx%
\newdimen\ancy%
\pgfextractx{\ancx}{\pgfpointorigin}%
\pgfextracty{\ancy}{\pgfpointorigin}%
\advance\ancy by 0.25cm%
\pgfpoint{\ancx}{\ancy}%
}
\savedanchor\southpoint{%
\newdimen\ancx%
\newdimen\ancy%
\pgfextractx{\ancx}{\pgfpointorigin}%
\pgfextracty{\ancy}{\pgfpointorigin}%
\advance\ancy by -0.25cm%
\pgfpoint{\ancx}{\ancy}%
}
\savedanchor\eastpoint{%
\newdimen\ancx%
\newdimen\ancy%
\pgfextractx{\ancx}{\pgfpointorigin}%
\pgfextracty{\ancy}{\pgfpointorigin}%
\advance\ancx by 0.25cm%
\pgfpoint{\ancx}{\ancy}%
}
\savedanchor\westpoint{%
\newdimen\ancx%
\newdimen\ancy%
\pgfextractx{\ancx}{\pgfpointorigin}%
\pgfextracty{\ancy}{\pgfpointorigin}%
\advance\ancx by -0.25cm%
\pgfpoint{\ancx}{\ancy}%
}
\anchor{center}{\pgfpointorigin}
\anchor{text}{\textpoint}
\anchor{east}{\eastpoint}
\anchor{west}{\westpoint}
\anchor{north}{\northpoint}
\anchor{south}{\southpoint}
\anchorborder{%
\newdimen\@tempdimxa
\newdimen\@tempdimya
\@tempdimxa=\pgf@x
\@tempdimya=\pgf@y
\pgfpointborderellipse{\pgfpoint{\@tempdimxa}{\@tempdimya}}
{\pgfpoint{0.25cm}{0.25cm}}
}
%%
\backgroundpath{
\pgfpathcircle{\pgfpointorigin}{0.25cm}%
%    \end{macrocode}
% get the current color to make sure that the text in the center  matches the 
% desired stroke color
%    \begin{macrocode}
\colorlet{saved}{.}
\pgftext{\color{saved} V}%
\pgfusepath{stroke}
}
}
%    \end{macrocode}
% \end{macro}
%
% \begin{macro}{switchtwoH}
%
%
%%%%%%%%%%%%%%%%%%%%%%%%%%%%%%%%%%%%%%%%%%%%%%%%%%%%%%%%%%%%%%%%%%%%%%%%%%%%%%%%%%%%%%%%%%%%%%%%%%%%%%%%%%%%%%%%%%%%%%%%%%%%%%%%%%%%%%%%%%%%%%%%%%%%%%%%%%%%%%%%%%%%%
%%
%% Shape: SPST switch (H)
%%
%%%%%%%%%%%%%%%%%%%%%%%%%%%%%%%%%%%%%%%%%%%%%%%%%%%%%%%%%%%%%%%%%%%%%%%%%%%%%%%%%%%%%%%%%%%%%%%%%%%%%%%%%%%%%%%%%%%%%%%%%%%%%%%%%%%%%%%%%%%%%%%%%%%%%%%%%%%%%%%%%%%%%
%
%
%
% The |switchtwoH| shape is a two position switch with oriented horizontally.
%
%    \begin{macrocode}
\pgfdeclareshape{switchtwoH}{
\savedanchor\textpoint{%
\newdimen\ancx
\newdimen\ancy
\pgfextractx{\ancx}{\pgfpointorigin}%
\pgfextracty{\ancy}{\pgfpointorigin}%
\advance\ancx by -0.5\wd\pgfnodeparttextbox%
\advance\ancy by -0.10cm
\advance\ancy by -\ht\pgfnodeparttextbox%
\pgfpoint{\ancx}{\ancy}%
}
\savedanchor\eastpoint{%
\newdimen\ancx%
\newdimen\ancy%
\pgfextractx{\ancx}{\pgfpointorigin}%
\pgfextracty{\ancy}{\pgfpointorigin}%
\advance\ancx by 0.30cm%
\pgfpoint{\ancx}{\ancy}%
}
\savedanchor\westpoint{%
\newdimen\ancx%
\newdimen\ancy%
\pgfextractx{\ancx}{\pgfpointorigin}%
\pgfextracty{\ancy}{\pgfpointorigin}%
\advance\ancx by -0.30cm%
\pgfpoint{\ancx}{\ancy}%
}
\anchor{center}{\pgfpointorigin}
\anchor{east}{\eastpoint}
\anchor{west}{\westpoint}
\anchor{text}{\textpoint}
\anchorborder{%
\newdimen\@tempdimxa
\newdimen\@tempdimya
\@tempdimxa=\pgf@x
\@tempdimya=\pgf@y
\pgfpointborderrectangle{\pgfpoint{\@tempdimxa}{\@tempdimya}}
{\pgfpoint{0.30cm}{0.10cm}}
}
\backgroundpath{
%% left contact
\pgfpathcircle{\pgfpoint{-0.25cm}{0.00cm}}{0.05cm}%
%% right contact
\pgfpathcircle{\pgfpoint{0.25cm}{0.00cm}}{0.05cm}%
%% switch
\pgfpathmoveto{\pgfpoint{-0.20cm}{0.00cm}}
\pgfpathlineto{\pgfpoint{0.2cm}{0.10cm}}
\pgfusepath{stroke}
}
}
%    \end{macrocode}
% \end{macro}
%
% \begin{macro}{switchtwoV}
%
%
%%%%%%%%%%%%%%%%%%%%%%%%%%%%%%%%%%%%%%%%%%%%%%%%%%%%%%%%%%%%%%%%%%%%%%%%%%%%%%%%%%%%%%%%%%%%%%%%%%%%%%%%%%%%%%%%%%%%%%%%%%%%%%%%%%%%%%%%%%%%%%%%%%%%%%%%%%%%%%%%%%%%%
%%
%% Shape: SPST switch (V)
%%
%%%%%%%%%%%%%%%%%%%%%%%%%%%%%%%%%%%%%%%%%%%%%%%%%%%%%%%%%%%%%%%%%%%%%%%%%%%%%%%%%%%%%%%%%%%%%%%%%%%%%%%%%%%%%%%%%%%%%%%%%%%%%%%%%%%%%%%%%%%%%%%%%%%%%%%%%%%%%%%%%%%%%
%
%
%
% The |switchtwoV| shape is a two position switch with oriented vertically.
%
%    \begin{macrocode}
\pgfdeclareshape{switchtwoV}{
\savedanchor\textpoint{%
\newdimen\ancx
\newdimen\ancy
\pgfextractx{\ancx}{\pgfpointorigin}%
\pgfextracty{\ancy}{\pgfpointorigin}%
\advance\ancx by 0.10cm
\advance\ancy by -0.5\ht\pgfnodeparttextbox%
\pgfpoint{\ancx}{\ancy}%
}
\savedanchor\northpoint{%
\newdimen\ancx%
\newdimen\ancy%
\pgfextractx{\ancx}{\pgfpointorigin}%
\pgfextracty{\ancy}{\pgfpointorigin}%
\advance\ancy by 0.30cm%
\pgfpoint{\ancx}{\ancy}%
}
\savedanchor\southpoint{%
\newdimen\ancx%
\newdimen\ancy%
\pgfextractx{\ancx}{\pgfpointorigin}%
\pgfextracty{\ancy}{\pgfpointorigin}%
\advance\ancy by -0.30cm%
\pgfpoint{\ancx}{\ancy}%
}
\anchor{center}{\pgfpointorigin}
\anchor{north}{\northpoint}
\anchor{south}{\southpoint}
\anchor{text}{\textpoint}
\anchorborder{%
\newdimen\@tempdimxa
\newdimen\@tempdimya
\@tempdimxa=\pgf@x
\@tempdimya=\pgf@y
\pgfpointborderrectangle{\pgfpoint{\@tempdimxa}{\@tempdimya}}
{\pgfpoint{0.10cm}{0.30cm}}
}
\backgroundpath{
%% bottom contact
\pgfpathcircle{\pgfpoint{0.0cm}{-0.25cm}}{0.05cm}%
%% top contact
\pgfpathcircle{\pgfpoint{0.0cm}{0.25cm}}{0.05cm}%
%% switch
\pgfpathmoveto{\pgfpoint{0.0cm}{-0.20cm}}
\pgfpathlineto{\pgfpoint{-0.1cm}{0.2cm}}
\pgfusepath{stroke}
}
}
%    \end{macrocode}
% \end{macro}
%
% \begin{macro}{switchthreeVDr}
%
%
%%%%%%%%%%%%%%%%%%%%%%%%%%%%%%%%%%%%%%%%%%%%%%%%%%%%%%%%%%%%%%%%%%%%%%%%%%%%%%%%%%%%%%%%%%%%%%%%%%%%%%%%%%%%%%%%%%%%%%%%%%%%%%%%%%%%%%%%%%%%%%%%%%%%%%%%%%%%%%%%%%%%%
%%
%% Shape: SPDT switch (V,D,r)
%%
%%%%%%%%%%%%%%%%%%%%%%%%%%%%%%%%%%%%%%%%%%%%%%%%%%%%%%%%%%%%%%%%%%%%%%%%%%%%%%%%%%%%%%%%%%%%%%%%%%%%%%%%%%%%%%%%%%%%%%%%%%%%%%%%%%%%%%%%%%%%%%%%%%%%%%%%%%%%%%%%%%%%%
%
%
%
% The |switchthreeVDr| shape is a switch with three contacts, oriented vertically, with single contact at top and switch to the right
%
%    \begin{macrocode}
\pgfdeclareshape{switchthreeVDr}{
\savedanchor\textpoint{%
\newdimen\ancx
\newdimen\ancy
\pgfextractx{\ancx}{\pgfpointorigin}%
\pgfextracty{\ancy}{\pgfpointorigin}%
\advance\ancx by 0.25cm
\advance\ancy by 0.125cm
\advance\ancy by 0.5\ht\pgfnodeparttextbox%
\pgfpoint{\ancx}{\ancy}%
}
\savedanchor\northpoint{%
\newdimen\ancx%
\newdimen\ancy%
\pgfextractx{\ancx}{\pgfpointorigin}%
\pgfextracty{\ancy}{\pgfpointorigin}%
\advance\ancy by 0.30cm%
\pgfpoint{\ancx}{\ancy}%
}
\savedanchor\southeastpoint{%
\newdimen\ancx%
\newdimen\ancy%
\pgfextractx{\ancx}{\pgfpointorigin}%
\pgfextracty{\ancy}{\pgfpointorigin}%
\advance\ancy by -0.05cm%
\advance\ancx by 0.25cm%
\pgfpoint{\ancx}{\ancy}%
}
\savedanchor\southwestpoint{%
\newdimen\ancx%
\newdimen\ancy%
\pgfextractx{\ancx}{\pgfpointorigin}%
\pgfextracty{\ancy}{\pgfpointorigin}%
\advance\ancy by -0.05cm%
\advance\ancx by -0.25cm%
\pgfpoint{\ancx}{\ancy}%
}
\anchor{center}{\pgfpointorigin}
\anchor{north}{\northpoint}
\anchor{southeast}{\southeastpoint}
\anchor{southwest}{\southwestpoint}
\anchor{text}{\textpoint}
\anchorborder{%
\newdimen\@tempdimxa
\newdimen\@tempdimya
\@tempdimxa=\pgf@x
\@tempdimya=\pgf@y
\pgfpointborderrectangle{\pgfpoint{\@tempdimxa}{\@tempdimya}}
{\pgfpoint{0.30cm}{0.30cm}} 
%    \end{macrocode}
% \emph{TODO}: need to better calculate the border
%    \begin{macrocode}
}
\backgroundpath{
%% top left contact
\pgfpathcircle{\pgfpoint{-0.25cm}{0.0cm}}{0.05cm}%
%% top right contact
\pgfpathcircle{\pgfpoint{0.25cm}{0.0cm}}{0.05cm}%
%% top contact
\pgfpathcircle{\pgfpoint{0.0cm}{0.25cm}}{0.05cm}%
%% switch
\pgfpathmoveto{\pgfpoint{0.035cm}{0.215cm}}
\pgfpathlineto{\pgfpoint{0.215cm}{0.035cm}}
\pgfusepath{stroke}
}
}
%    \end{macrocode}
% \end{macro}
%
% \begin{macro}{switchthreeVDl}
%
%
%%%%%%%%%%%%%%%%%%%%%%%%%%%%%%%%%%%%%%%%%%%%%%%%%%%%%%%%%%%%%%%%%%%%%%%%%%%%%%%%%%%%%%%%%%%%%%%%%%%%%%%%%%%%%%%%%%%%%%%%%%%%%%%%%%%%%%%%%%%%%%%%%%%%%%%%%%%%%%%%%%%%%
%%
%% Shape: SPDT switch (V,D,l)
%%
%%%%%%%%%%%%%%%%%%%%%%%%%%%%%%%%%%%%%%%%%%%%%%%%%%%%%%%%%%%%%%%%%%%%%%%%%%%%%%%%%%%%%%%%%%%%%%%%%%%%%%%%%%%%%%%%%%%%%%%%%%%%%%%%%%%%%%%%%%%%%%%%%%%%%%%%%%%%%%%%%%%%%
%
%
%
% The |switchthreeVDl| shape is a switch with three contacts, oriented 
% vertically, with single contact at top and switch to the left.
%
%    \begin{macrocode}
\pgfdeclareshape{switchthreeVDl}{
\savedanchor\textpoint{%
\newdimen\ancx
\newdimen\ancy
\pgfextractx{\ancx}{\pgfpointorigin}%
\pgfextracty{\ancy}{\pgfpointorigin}%
\advance\ancx by 0.25cm
\advance\ancy by 0.125cm
\advance\ancy by 0.5\ht\pgfnodeparttextbox%
\pgfpoint{\ancx}{\ancy}%
}
\savedanchor\northpoint{%
\newdimen\ancx%
\newdimen\ancy%
\pgfextractx{\ancx}{\pgfpointorigin}%
\pgfextracty{\ancy}{\pgfpointorigin}%
\advance\ancy by 0.30cm%
\pgfpoint{\ancx}{\ancy}%
}
\savedanchor\southeastpoint{%
\newdimen\ancx%
\newdimen\ancy%
\pgfextractx{\ancx}{\pgfpointorigin}%
\pgfextracty{\ancy}{\pgfpointorigin}%
\advance\ancy by -0.05cm%
\advance\ancx by 0.25cm%
\pgfpoint{\ancx}{\ancy}%
}
\savedanchor\southwestpoint{%
\newdimen\ancx%
\newdimen\ancy%
\pgfextractx{\ancx}{\pgfpointorigin}%
\pgfextracty{\ancy}{\pgfpointorigin}%
\advance\ancy by -0.05cm%
\advance\ancx by -0.25cm%
\pgfpoint{\ancx}{\ancy}%
}
\anchor{center}{\pgfpointorigin}
\anchor{north}{\northpoint}
\anchor{southeast}{\southeastpoint}
\anchor{southwest}{\southwestpoint}
\anchor{text}{\textpoint}
\anchorborder{%
\newdimen\@tempdimxa
\newdimen\@tempdimya
\@tempdimxa=\pgf@x
\@tempdimya=\pgf@y
\pgfpointborderrectangle{\pgfpoint{\@tempdimxa}{\@tempdimya}}
{\pgfpoint{0.30cm}{0.30cm}} 
%    \end{macrocode}
% \emph{TODO}: need to better calculate the border
%    \begin{macrocode}
}
\backgroundpath{
%% left
\pgfpathcircle{\pgfpoint{-0.25cm}{0.0cm}}{0.05cm}%
%% right
\pgfpathcircle{\pgfpoint{0.25cm}{0.0cm}}{0.05cm}%
%% top
\pgfpathcircle{\pgfpoint{0.0cm}{0.25cm}}{0.05cm}%
%% switch
\pgfpathmoveto{\pgfpoint{-0.035cm}{0.215cm}}
\pgfpathlineto{\pgfpoint{-0.215cm}{0.035cm}}
\pgfusepath{stroke}
}
}
%    \end{macrocode}
% \end{macro}
%
% \begin{macro}{switchthreeVUr}
%
%
%%%%%%%%%%%%%%%%%%%%%%%%%%%%%%%%%%%%%%%%%%%%%%%%%%%%%%%%%%%%%%%%%%%%%%%%%%%%%%%%%%%%%%%%%%%%%%%%%%%%%%%%%%%%%%%%%%%%%%%%%%%%%%%%%%%%%%%%%%%%%%%%%%%%%%%%%%%%%%%%%%%%%
%%
%% Shape: SPDT switch (V,U,r)
%%
%%%%%%%%%%%%%%%%%%%%%%%%%%%%%%%%%%%%%%%%%%%%%%%%%%%%%%%%%%%%%%%%%%%%%%%%%%%%%%%%%%%%%%%%%%%%%%%%%%%%%%%%%%%%%%%%%%%%%%%%%%%%%%%%%%%%%%%%%%%%%%%%%%%%%%%%%%%%%%%%%%%%%
%
%
%
% The |switchthreeVUr| shape is a switch with three contacts, oriented 
% vertically, with single contact at bottom and switch to the right.
%
%    \begin{macrocode}
\pgfdeclareshape{switchthreeVUr}{
\savedanchor\textpoint{%
\newdimen\ancx
\newdimen\ancy
\pgfextractx{\ancx}{\pgfpointorigin}%
\pgfextracty{\ancy}{\pgfpointorigin}%
\advance\ancx by 0.25cm
\advance\ancy by -0.125cm
\advance\ancy by -0.5\ht\pgfnodeparttextbox%
\pgfpoint{\ancx}{\ancy}%
}
\savedanchor\southpoint{%
\newdimen\ancx%
\newdimen\ancy%
\pgfextractx{\ancx}{\pgfpointorigin}%
\pgfextracty{\ancy}{\pgfpointorigin}%
\advance\ancy by -0.30cm%
\pgfpoint{\ancx}{\ancy}%
}
\savedanchor\northeastpoint{%
\newdimen\ancx%
\newdimen\ancy%
\pgfextractx{\ancx}{\pgfpointorigin}%
\pgfextracty{\ancy}{\pgfpointorigin}%
\advance\ancy by 0.05cm%
\advance\ancx by 0.25cm%
\pgfpoint{\ancx}{\ancy}%
}
\savedanchor\northwestpoint{%
\newdimen\ancx%
\newdimen\ancy%
\pgfextractx{\ancx}{\pgfpointorigin}%
\pgfextracty{\ancy}{\pgfpointorigin}%
\advance\ancy by 0.05cm%
\advance\ancx by -0.25cm%
\pgfpoint{\ancx}{\ancy}%
}
\anchor{center}{\pgfpointorigin}
\anchor{south}{\southpoint}
\anchor{northeast}{\northeastpoint}
\anchor{northwest}{\northwestpoint}
\anchor{text}{\textpoint}
\anchorborder{%
\newdimen\@tempdimxa
\newdimen\@tempdimya
\@tempdimxa=\pgf@x
\@tempdimya=\pgf@y
\pgfpointborderrectangle{\pgfpoint{\@tempdimxa}{\@tempdimya}}
{\pgfpoint{0.30cm}{0.30cm}} 
%    \end{macrocode}
% \emph{TODO}: need to better calculate the border
%    \begin{macrocode}
}
\backgroundpath{
%% left
\pgfpathcircle{\pgfpoint{-0.25cm}{0.0cm}}{0.05cm}%
%% right
\pgfpathcircle{\pgfpoint{0.25cm}{0.0cm}}{0.05cm}%
%% bottom
\pgfpathcircle{\pgfpoint{0.0cm}{-0.25cm}}{0.05cm}%
%% switch
\pgfpathmoveto{\pgfpoint{0.035cm}{-0.215cm}}
\pgfpathlineto{\pgfpoint{0.215cm}{-0.035cm}}
\pgfusepath{stroke}
}
}
%    \end{macrocode}
% \end{macro}
%
% \begin{macro}{switchthreeVUl}
%
%
%%%%%%%%%%%%%%%%%%%%%%%%%%%%%%%%%%%%%%%%%%%%%%%%%%%%%%%%%%%%%%%%%%%%%%%%%%%%%%%%%%%%%%%%%%%%%%%%%%%%%%%%%%%%%%%%%%%%%%%%%%%%%%%%%%%%%%%%%%%%%%%%%%%%%%%%%%%%%%%%%%%%%
%%
%% Shape: SPDT switch (V,U,l)
%%
%%%%%%%%%%%%%%%%%%%%%%%%%%%%%%%%%%%%%%%%%%%%%%%%%%%%%%%%%%%%%%%%%%%%%%%%%%%%%%%%%%%%%%%%%%%%%%%%%%%%%%%%%%%%%%%%%%%%%%%%%%%%%%%%%%%%%%%%%%%%%%%%%%%%%%%%%%%%%%%%%%%%%
%
%
%
% The |switchthreeVUl| shape is a switch with three contacts, oriented 
% vertically, with single contact at bottom and switch to the left.
%
%    \begin{macrocode}
\pgfdeclareshape{switchthreeVUl}{
\savedanchor\textpoint{%
\newdimen\ancx
\newdimen\ancy
\pgfextractx{\ancx}{\pgfpointorigin}%
\pgfextracty{\ancy}{\pgfpointorigin}%
\advance\ancx by 0.25cm
\advance\ancy by -0.125cm
\advance\ancy by -0.5\ht\pgfnodeparttextbox%
\pgfpoint{\ancx}{\ancy}%
}
\savedanchor\southpoint{%
\newdimen\ancx%
\newdimen\ancy%
\pgfextractx{\ancx}{\pgfpointorigin}%
\pgfextracty{\ancy}{\pgfpointorigin}%
\advance\ancy by -0.30cm%
\pgfpoint{\ancx}{\ancy}%
}
\savedanchor\northeastpoint{%
\newdimen\ancx%
\newdimen\ancy%
\pgfextractx{\ancx}{\pgfpointorigin}%
\pgfextracty{\ancy}{\pgfpointorigin}%
\advance\ancy by 0.05cm%
\advance\ancx by 0.25cm%
\pgfpoint{\ancx}{\ancy}%
}
\savedanchor\northwestpoint{%
\newdimen\ancx%
\newdimen\ancy%
\pgfextractx{\ancx}{\pgfpointorigin}%
\pgfextracty{\ancy}{\pgfpointorigin}%
\advance\ancy by 0.05cm%
\advance\ancx by -0.25cm%
\pgfpoint{\ancx}{\ancy}%
}
\anchor{center}{\pgfpointorigin}
\anchor{south}{\southpoint}
\anchor{northeast}{\northeastpoint}
\anchor{northwest}{\northwestpoint}
\anchor{text}{\textpoint}
\anchorborder{%
\newdimen\@tempdimxa
\newdimen\@tempdimya
\@tempdimxa=\pgf@x
\@tempdimya=\pgf@y
\pgfpointborderrectangle{\pgfpoint{\@tempdimxa}{\@tempdimya}}
{\pgfpoint{0.30cm}{0.30cm}} 
%    \end{macrocode}
% \emph{TODO}: need to better calculate the border
%    \begin{macrocode}
}
\backgroundpath{
%% left
\pgfpathcircle{\pgfpoint{-0.25cm}{0.0cm}}{0.05cm}%
%% right
\pgfpathcircle{\pgfpoint{0.25cm}{0.0cm}}{0.05cm}%
%% bottom
\pgfpathcircle{\pgfpoint{0.0cm}{-0.25cm}}{0.05cm}%
%% switch
\pgfpathmoveto{\pgfpoint{-0.035cm}{-0.215cm}}
\pgfpathlineto{\pgfpoint{-0.215cm}{-0.035cm}}
\pgfusepath{stroke}
}
}
%    \end{macrocode}
% \end{macro}
%
% \begin{macro}{switchthreeHRu}
%
%
%%%%%%%%%%%%%%%%%%%%%%%%%%%%%%%%%%%%%%%%%%%%%%%%%%%%%%%%%%%%%%%%%%%%%%%%%%%%%%%%%%%%%%%%%%%%%%%%%%%%%%%%%%%%%%%%%%%%%%%%%%%%%%%%%%%%%%%%%%%%%%%%%%%%%%%%%%%%%%%%%%%%%
%%
%% Shape: SPDT switch (H,R,u)
%%
%%%%%%%%%%%%%%%%%%%%%%%%%%%%%%%%%%%%%%%%%%%%%%%%%%%%%%%%%%%%%%%%%%%%%%%%%%%%%%%%%%%%%%%%%%%%%%%%%%%%%%%%%%%%%%%%%%%%%%%%%%%%%%%%%%%%%%%%%%%%%%%%%%%%%%%%%%%%%%%%%%%%%
%
%
%
% The |switchthreeHRu| shape is a switch with three contacts, oriented 
% horizontally, with single contact at left and switch upward.
%
%    \begin{macrocode}
\pgfdeclareshape{switchthreeHRu}{
\savedanchor\textpoint{%
\newdimen\ancx
\newdimen\ancy
\pgfextractx{\ancx}{\pgfpointorigin}%
\pgfextracty{\ancy}{\pgfpointorigin}%
\advance\ancx by -0.25cm
\advance\ancx by -\wd\pgfnodeparttextbox
\advance\ancy by -0.25cm
\advance\ancy by -0.5\ht\pgfnodeparttextbox%
\pgfpoint{\ancx}{\ancy}%
}
\savedanchor\westpoint{%
\newdimen\ancx%
\newdimen\ancy%
\pgfextractx{\ancx}{\pgfpointorigin}%
\pgfextracty{\ancy}{\pgfpointorigin}%
\advance\ancx by -0.30cm%
\pgfpoint{\ancx}{\ancy}%
}
\savedanchor\northeastpoint{%
\newdimen\ancx%
\newdimen\ancy%
\pgfextractx{\ancx}{\pgfpointorigin}%
\pgfextracty{\ancy}{\pgfpointorigin}%
\advance\ancx by 0.05cm%
\advance\ancy by 0.25cm%
\pgfpoint{\ancx}{\ancy}%
}
\savedanchor\southeastpoint{%
\newdimen\ancx%
\newdimen\ancy%
\pgfextractx{\ancx}{\pgfpointorigin}%
\pgfextracty{\ancy}{\pgfpointorigin}%
\advance\ancx by 0.05cm%
\advance\ancy by -0.25cm%
\pgfpoint{\ancx}{\ancy}%
}
\anchor{center}{\pgfpointorigin}
\anchor{west}{\westpoint}
\anchor{northeast}{\northeastpoint}
\anchor{southeast}{\southeastpoint}
\anchor{text}{\textpoint}
\anchorborder{%
\newdimen\@tempdimxa
\newdimen\@tempdimya
\@tempdimxa=\pgf@x
\@tempdimya=\pgf@y
\pgfpointborderrectangle{\pgfpoint{\@tempdimxa}{\@tempdimya}}
{\pgfpoint{0.30cm}{0.30cm}} 
%    \end{macrocode}
% \emph{TODO}: need to better calculate the border
%    \begin{macrocode}
}
\backgroundpath{
%% left
\pgfpathcircle{\pgfpoint{-0.25cm}{0.0cm}}{0.05cm}%
%% bottom
\pgfpathcircle{\pgfpoint{0.0cm}{-0.25cm}}{0.05cm}%
%% top
\pgfpathcircle{\pgfpoint{0.0cm}{0.25cm}}{0.05cm}%
%% switch
\pgfpathmoveto{\pgfpoint{-0.035cm}{0.215cm}}
\pgfpathlineto{\pgfpoint{-0.215cm}{0.035cm}}
\pgfusepath{stroke}
}
}
%    \end{macrocode}
% \end{macro}
%
% \begin{macro}{switchthreeHRd}
%
%
%%%%%%%%%%%%%%%%%%%%%%%%%%%%%%%%%%%%%%%%%%%%%%%%%%%%%%%%%%%%%%%%%%%%%%%%%%%%%%%%%%%%%%%%%%%%%%%%%%%%%%%%%%%%%%%%%%%%%%%%%%%%%%%%%%%%%%%%%%%%%%%%%%%%%%%%%%%%%%%%%%%%%
%%
%% Shape: SPDT switch (H,R,d)
%%
%%%%%%%%%%%%%%%%%%%%%%%%%%%%%%%%%%%%%%%%%%%%%%%%%%%%%%%%%%%%%%%%%%%%%%%%%%%%%%%%%%%%%%%%%%%%%%%%%%%%%%%%%%%%%%%%%%%%%%%%%%%%%%%%%%%%%%%%%%%%%%%%%%%%%%%%%%%%%%%%%%%%%
%
%
%
% The |switchthreeHRd| shape is a switch with three contacts, oriented \
% horizontally, with single contact at left and switch downward.
%
%    \begin{macrocode}
\pgfdeclareshape{switchthreeHRd}{
\savedanchor\textpoint{%
\newdimen\ancx
\newdimen\ancy
\pgfextractx{\ancx}{\pgfpointorigin}%
\pgfextracty{\ancy}{\pgfpointorigin}%
\advance\ancx by -0.25cm
\advance\ancx by -\wd\pgfnodeparttextbox
\advance\ancy by -0.25cm
\advance\ancy by -0.5\ht\pgfnodeparttextbox%
\pgfpoint{\ancx}{\ancy}%
}
\savedanchor\westpoint{%
\newdimen\ancx%
\newdimen\ancy%
\pgfextractx{\ancx}{\pgfpointorigin}%
\pgfextracty{\ancy}{\pgfpointorigin}%
\advance\ancx by -0.30cm%
\pgfpoint{\ancx}{\ancy}%
}
\savedanchor\northeastpoint{%
\newdimen\ancx%
\newdimen\ancy%
\pgfextractx{\ancx}{\pgfpointorigin}%
\pgfextracty{\ancy}{\pgfpointorigin}%
\advance\ancx by 0.05cm%
\advance\ancy by 0.25cm%
\pgfpoint{\ancx}{\ancy}%
}
\savedanchor\southeastpoint{%
\newdimen\ancx%
\newdimen\ancy%
\pgfextractx{\ancx}{\pgfpointorigin}%
\pgfextracty{\ancy}{\pgfpointorigin}%
\advance\ancx by 0.05cm%
\advance\ancy by -0.25cm%
\pgfpoint{\ancx}{\ancy}%
}
\anchor{center}{\pgfpointorigin}
\anchor{west}{\westpoint}
\anchor{northeast}{\northeastpoint}
\anchor{southeast}{\southeastpoint}
\anchor{text}{\textpoint}
\anchorborder{%
\newdimen\@tempdimxa
\newdimen\@tempdimya
\@tempdimxa=\pgf@x
\@tempdimya=\pgf@y
\pgfpointborderrectangle{\pgfpoint{\@tempdimxa}{\@tempdimya}}
{\pgfpoint{0.30cm}{0.30cm}} 
%    \end{macrocode}
% \emph{TODO}: need to better calculate the border
%    \begin{macrocode}
}
\backgroundpath{
%%
\pgfpathcircle{\pgfpoint{-0.25cm}{0.0cm}}{0.05cm}%
%%
\pgfpathcircle{\pgfpoint{0.0cm}{-0.25cm}}{0.05cm}%
%%
\pgfpathcircle{\pgfpoint{0.0cm}{0.25cm}}{0.05cm}%
%%
\pgfpathmoveto{\pgfpoint{-0.035cm}{-0.215cm}}
\pgfpathlineto{\pgfpoint{-0.215cm}{-0.035cm}}
\pgfusepath{stroke}
}
}
%    \end{macrocode}
% \end{macro}
%
% \begin{macro}{switchthreeHLu}
%
%
%%%%%%%%%%%%%%%%%%%%%%%%%%%%%%%%%%%%%%%%%%%%%%%%%%%%%%%%%%%%%%%%%%%%%%%%%%%%%%%%%%%%%%%%%%%%%%%%%%%%%%%%%%%%%%%%%%%%%%%%%%%%%%%%%%%%%%%%%%%%%%%%%%%%%%%%%%%%%%%%%%%%%
%%
%% Shape: SPDT switch (H,L,u)
%%
%%%%%%%%%%%%%%%%%%%%%%%%%%%%%%%%%%%%%%%%%%%%%%%%%%%%%%%%%%%%%%%%%%%%%%%%%%%%%%%%%%%%%%%%%%%%%%%%%%%%%%%%%%%%%%%%%%%%%%%%%%%%%%%%%%%%%%%%%%%%%%%%%%%%%%%%%%%%%%%%%%%%%
%
%
%
% The |switchthreeHLu| shape is a switch with three contacts, oriented 
% horizontally, with single contact at right and switch upward.
%
%    \begin{macrocode}
\pgfdeclareshape{switchthreeHLu}{
\savedanchor\textpoint{%
\newdimen\ancx
\newdimen\ancy
\pgfextractx{\ancx}{\pgfpointorigin}%
\pgfextracty{\ancy}{\pgfpointorigin}%
\advance\ancx by 0.25cm
\advance\ancy by -0.25cm
\advance\ancy by -0.5\ht\pgfnodeparttextbox%
\pgfpoint{\ancx}{\ancy}%
}
\savedanchor\eastpoint{%
\newdimen\ancx%
\newdimen\ancy%
\pgfextractx{\ancx}{\pgfpointorigin}%
\pgfextracty{\ancy}{\pgfpointorigin}%
\advance\ancx by 0.30cm%
\pgfpoint{\ancx}{\ancy}%
}
\savedanchor\northwestpoint{%
\newdimen\ancx%
\newdimen\ancy%
\pgfextractx{\ancx}{\pgfpointorigin}%
\pgfextracty{\ancy}{\pgfpointorigin}%
\advance\ancx by -0.05cm%
\advance\ancy by 0.25cm%
\pgfpoint{\ancx}{\ancy}%
}
\savedanchor\southwestpoint{%
\newdimen\ancx%
\newdimen\ancy%
\pgfextractx{\ancx}{\pgfpointorigin}%
\pgfextracty{\ancy}{\pgfpointorigin}%
\advance\ancx by -0.05cm%
\advance\ancy by -0.25cm%
\pgfpoint{\ancx}{\ancy}%
}
\anchor{center}{\pgfpointorigin}
\anchor{east}{\eastpoint}
\anchor{northwest}{\northwestpoint}
\anchor{southwest}{\southwestpoint}
\anchor{text}{\textpoint}
\anchorborder{%
\newdimen\@tempdimxa
\newdimen\@tempdimya
\@tempdimxa=\pgf@x
\@tempdimya=\pgf@y
\pgfpointborderrectangle{\pgfpoint{\@tempdimxa}{\@tempdimya}}
{\pgfpoint{0.30cm}{0.30cm}} 
%    \end{macrocode}
% \emph{TODO}: need to better calculate the border
%    \begin{macrocode}
}
\backgroundpath{
%% right
\pgfpathcircle{\pgfpoint{0.25cm}{0.0cm}}{0.05cm}%
%% bottom
\pgfpathcircle{\pgfpoint{0.0cm}{-0.25cm}}{0.05cm}%
%% top
\pgfpathcircle{\pgfpoint{0.0cm}{0.25cm}}{0.05cm}%
%% switch
\pgfpathmoveto{\pgfpoint{0.035cm}{0.215cm}}
\pgfpathlineto{\pgfpoint{0.215cm}{0.035cm}}
\pgfusepath{stroke}
}
}
%    \end{macrocode}
% \end{macro}
%
% \begin{macro}{switchthreeHLd}
%
%
%%%%%%%%%%%%%%%%%%%%%%%%%%%%%%%%%%%%%%%%%%%%%%%%%%%%%%%%%%%%%%%%%%%%%%%%%%%%%%%%%%%%%%%%%%%%%%%%%%%%%%%%%%%%%%%%%%%%%%%%%%%%%%%%%%%%%%%%%%%%%%%%%%%%%%%%%%%%%%%%%%%%%
%%
%% Shape: SPDT switch (H,L,d)
%%
%%%%%%%%%%%%%%%%%%%%%%%%%%%%%%%%%%%%%%%%%%%%%%%%%%%%%%%%%%%%%%%%%%%%%%%%%%%%%%%%%%%%%%%%%%%%%%%%%%%%%%%%%%%%%%%%%%%%%%%%%%%%%%%%%%%%%%%%%%%%%%%%%%%%%%%%%%%%%%%%%%%%%
%
%
%
% The |switchthreeHLd| shape is a switch with three contacts, oriented 
% horizontally, with single contact at right and switch downward.
%
%    \begin{macrocode}
\pgfdeclareshape{switchthreeHLd}{
\savedanchor\textpoint{%
\newdimen\ancx
\newdimen\ancy
\pgfextractx{\ancx}{\pgfpointorigin}%
\pgfextracty{\ancy}{\pgfpointorigin}%
\advance\ancx by 0.25cm
\advance\ancy by -0.25cm
\advance\ancy by -0.5\ht\pgfnodeparttextbox%
\pgfpoint{\ancx}{\ancy}%
}
\savedanchor\eastpoint{%
\newdimen\ancx%
\newdimen\ancy%
\pgfextractx{\ancx}{\pgfpointorigin}%
\pgfextracty{\ancy}{\pgfpointorigin}%
\advance\ancx by 0.30cm%
\pgfpoint{\ancx}{\ancy}%
}
\savedanchor\northwestpoint{%
\newdimen\ancx%
\newdimen\ancy%
\pgfextractx{\ancx}{\pgfpointorigin}%
\pgfextracty{\ancy}{\pgfpointorigin}%
\advance\ancx by -0.05cm%
\advance\ancy by 0.25cm%
\pgfpoint{\ancx}{\ancy}%
}
\savedanchor\southwestpoint{%
\newdimen\ancx%
\newdimen\ancy%
\pgfextractx{\ancx}{\pgfpointorigin}%
\pgfextracty{\ancy}{\pgfpointorigin}%
\advance\ancx by -0.05cm%
\advance\ancy by -0.25cm%
\pgfpoint{\ancx}{\ancy}%
}
\anchor{center}{\pgfpointorigin}
\anchor{east}{\eastpoint}
\anchor{northwest}{\northwestpoint}
\anchor{southwest}{\southwestpoint}
\anchor{text}{\textpoint}
\anchorborder{%
\newdimen\@tempdimxa
\newdimen\@tempdimya
\@tempdimxa=\pgf@x
\@tempdimya=\pgf@y
\pgfpointborderrectangle{\pgfpoint{\@tempdimxa}{\@tempdimya}}
{\pgfpoint{0.30cm}{0.30cm}} 
%    \end{macrocode}
% \emph{TODO}: need to better calculate the border
%    \begin{macrocode}
}
\backgroundpath{
%% right
\pgfpathcircle{\pgfpoint{0.25cm}{0.0cm}}{0.05cm}%
%% bottom
\pgfpathcircle{\pgfpoint{0.0cm}{-0.25cm}}{0.05cm}%
%% top
\pgfpathcircle{\pgfpoint{0.0cm}{0.25cm}}{0.05cm}%
%% switch
\pgfpathmoveto{\pgfpoint{0.035cm}{-0.215cm}}
\pgfpathlineto{\pgfpoint{0.215cm}{-0.035cm}}
\pgfusepath{stroke}
}
}
%    \end{macrocode}
% \end{macro}
%
% \begin{macro}{vectIn}
%
%
%%%%%%%%%%%%%%%%%%%%%%%%%%%%%%%%%%%%%%%%%%%%%%%%%%%%%%%%%%%%%%%%%%%%%%%%%%%%%%%%%%%%%%%%%%%%%%%%%%%%%%%%%%%%%%%%%%%%%%%%%%%%%%%%%%%%%%%%%%%%%%%%%%%%%%%%%%%%%%%%%%%%%
%%
%% Shape: Inward Vector
%%
%%%%%%%%%%%%%%%%%%%%%%%%%%%%%%%%%%%%%%%%%%%%%%%%%%%%%%%%%%%%%%%%%%%%%%%%%%%%%%%%%%%%%%%%%%%%%%%%%%%%%%%%%%%%%%%%%%%%%%%%%%%%%%%%%%%%%%%%%%%%%%%%%%%%%%%%%%%%%%%%%%%%%
%
%
%
% The |vectIn| shape is a vector going into the page
%
%    \begin{macrocode}
\pgfdeclareshape{vectIn}{
\savedanchor\textpoint{%
\newdimen\ancx
\newdimen\ancy
\pgfextractx{\ancx}{\pgfpointorigin}%
\pgfextracty{\ancy}{\pgfpointorigin}%
\advance\ancx by 0.25cm
\advance\ancy by -0.25cm
\advance\ancy by -0.5\ht\pgfnodeparttextbox%
\pgfpoint{\ancx}{\ancy}%
}
\anchor{center}{\pgfpointorigin}
\anchor{text}{\textpoint}
\anchorborder{%
\newdimen\@tempdimxa
\newdimen\@tempdimya
\@tempdimxa=\pgf@x
\@tempdimya=\pgf@y
\pgfpointborderellipse{\pgfpoint{\@tempdimxa}{\@tempdimya}}
{\pgfpoint{0.1cm}{0.1cm}}
}
\backgroundpath{
%
\pgfpathcircle{\pgfpointorigin}{0.1cm}%
%
\pgfpathmoveto{\pgfpointpolar{45}{0.1cm}}
\pgfpathlineto{\pgfpointpolar{225}{0.1cm}}
%
\pgfpathmoveto{\pgfpointpolar{135}{0.1cm}}
\pgfpathlineto{\pgfpointpolar{315}{0.1cm}}
\pgfusepath{stroke}
}
}
%    \end{macrocode}
% \end{macro}
%
% \begin{macro}{vectIn}
%
%
%%%%%%%%%%%%%%%%%%%%%%%%%%%%%%%%%%%%%%%%%%%%%%%%%%%%%%%%%%%%%%%%%%%%%%%%%%%%%%%%%%%%%%%%%%%%%%%%%%%%%%%%%%%%%%%%%%%%%%%%%%%%%%%%%%%%%%%%%%%%%%%%%%%%%%%%%%%%%%%%%%%%%
%%
%% Shape: Outward Vector
%%
%%%%%%%%%%%%%%%%%%%%%%%%%%%%%%%%%%%%%%%%%%%%%%%%%%%%%%%%%%%%%%%%%%%%%%%%%%%%%%%%%%%%%%%%%%%%%%%%%%%%%%%%%%%%%%%%%%%%%%%%%%%%%%%%%%%%%%%%%%%%%%%%%%%%%%%%%%%%%%%%%%%%%
%
%
%
% The |vectOut| shape is a vector going out of the page
%
%    \begin{macrocode}
\pgfdeclareshape{vectOut}{
\savedanchor\textpoint{%
\newdimen\ancx
\newdimen\ancy
\pgfextractx{\ancx}{\pgfpointorigin}%
\pgfextracty{\ancy}{\pgfpointorigin}%
\advance\ancx by 0.25cm
\advance\ancy by -0.25cm
\advance\ancy by -0.5\ht\pgfnodeparttextbox%
\pgfpoint{\ancx}{\ancy}%
}
\anchor{center}{\pgfpointorigin}
\anchor{text}{\textpoint}
\anchorborder{%
\newdimen\@tempdimxa
\newdimen\@tempdimya
\@tempdimxa=\pgf@x
\@tempdimya=\pgf@y
\pgfpointborderellipse{\pgfpoint{\@tempdimxa}{\@tempdimya}}
{\pgfpoint{0.1cm}{0.1cm}}
}
\backgroundpath{
%
\pgfpathcircle{\pgfpointorigin}{0.1cm}%
\pgfusepath{stroke}%
%
\pgfpathcircle{\pgfpointorigin}{0.025cm}%
\pgfusepath{fill}%
}
}
%    \end{macrocode}
% \end{macro}
%
% \begin{macro}{lightbulb}
%
%
%%%%%%%%%%%%%%%%%%%%%%%%%%%%%%%%%%%%%%%%%%%%%%%%%%%%%%%%%%%%%%%%%%%%%%%%%%%%%%%%%%%%%%%%%%%%%%%%%%%%%%%%%%%%%%%%%%%%%%%%%%%%%%%%%%%%%%%%%%%%%%%%%%%%%%%%%%%%%%%%%%%%%
%%
%% Shape: Light bulb
%%
%%%%%%%%%%%%%%%%%%%%%%%%%%%%%%%%%%%%%%%%%%%%%%%%%%%%%%%%%%%%%%%%%%%%%%%%%%%%%%%%%%%%%%%%%%%%%%%%%%%%%%%%%%%%%%%%%%%%%%%%%%%%%%%%%%%%%%%%%%%%%%%%%%%%%%%%%%%%%%%%%%%%%
%
%
%
% The |lightbulb| shape is a line drawing of an incandescent light bulb
%
%    \begin{macrocode}
\pgfdeclareshape{lightbulb}{
\savedanchor\textpoint{%
\newdimen\ancx%
\newdimen\ancy%
\pgfextractx{\ancx}{\pgfpointorigin}%
\pgfextracty{\ancy}{\pgfpointorigin}%
\pgfpoint{\ancx}{\ancy}%
}
\anchor{center}{\pgfpointorigin}
\anchor{text}{\pgfpointorigin}
\anchorborder{%
\newdimen\@tempdimxa
\newdimen\@tempdimya
\@tempdimxa=\pgf@x
\@tempdimya=\pgf@y
\pgfpointborderrectangle{\pgfpoint{\@tempdimxa}{\@tempdimya}}
{\pgfpoint{1.5cm}{6.1cm}} %%@@TODO: fix border
}
\backgroundpath{
%% bulb
\pgfpathmoveto{\pgfpoint{-1.3cm}{-5.5cm}}
\pgfpatharc{270}{90}{0.35cm}
\pgfpatharc{270}{90}{0.35cm}
\pgfpatharc{270}{90}{0.35cm}
\pgfpatharc{270}{90}{0.35cm}
\pgfpatharc{244.3207}{-64.3207}{3.0cm}
\pgfpatharc{90}{-90}{0.35cm}
\pgfpatharc{90}{-90}{0.35cm}
\pgfpatharc{90}{-90}{0.35cm}
\pgfpatharc{90}{-90}{0.35cm}
\pgfpathclose
%%
\pgfpathmoveto{\pgfpoint{-1.3cm}{-5.5cm}}
\pgfpathlineto{\pgfpoint{1.3cm}{-4.8cm}}
%%
\pgfpathmoveto{\pgfpoint{-1.3cm}{-4.8cm}}
\pgfpathlineto{\pgfpoint{1.3cm}{-4.1cm}}
%%
\pgfpathmoveto{\pgfpoint{-1.3cm}{-4.1cm}}
\pgfpathlineto{\pgfpoint{1.3cm}{-3.4cm}}
%%
\pgfpathmoveto{\pgfpoint{-1.3cm}{-3.4cm}}
\pgfpathlineto{\pgfpoint{1.3cm}{-2.7cm}}
%%
\pgfpathmoveto{\pgfpoint{-1.3cm}{-2.7cm}}
\pgfpathlineto{\pgfpoint{1.3cm}{-2.7cm}}
%% filament?
\pgfpathmoveto{\pgfpoint{-0.383cm}{-2.7cm}}
\pgfpathlineto{\pgfpoint{-0.383cm}{-1.2835cm}}
\pgfpathlineto{\pgfpoint{-1.4665cm}{0cm}}
\pgfpathlineto{\pgfpoint{-1.4665cm}{0.2cm}}
\pgfpatharc{180}{0}{0.25cm}
\pgfpatharc{0}{-180}{0.10cm}%0.3
\pgfpatharc{180}{0}{0.25cm}
\pgfpatharc{0}{-180}{0.10cm}%0.6
\pgfpatharc{180}{0}{0.25cm}
\pgfpatharc{0}{-180}{0.10cm}%0.9
\pgfpatharc{180}{0}{0.25cm}
\pgfpatharc{0}{-180}{0.10cm}%1.2
\pgfpatharc{180}{0}{0.25cm}
\pgfpatharc{0}{-180}{0.10cm}%1.5
\pgfpatharc{180}{0}{0.25cm}
\pgfpatharc{0}{-180}{0.10cm}%1.8
\pgfpatharc{180}{0}{0.25cm}
\pgfpatharc{0}{-180}{0.10cm}%2.1
\pgfpatharc{180}{0}{0.25cm}
\pgfpatharc{0}{-180}{0.10cm}%2.4
\pgfpatharc{180}{0}{0.25cm}%2.9
\pgfpathlineto{\pgfpoint{1.4665cm}{0.0cm}}
\pgfpathlineto{\pgfpoint{+0.383cm}{-1.2835cm}}
\pgfpathlineto{\pgfpoint{+0.383cm}{-2.7cm}}
\pgfusepath{stroke}
%% base of filament
\pgfpathmoveto{\pgfpoint{-1.3cm}{-5.5cm}}
\pgfpathlineto{\pgfpoint{-0.2cm}{-6.1cm}}
\pgfpathlineto{\pgfpoint{+0.2cm}{-6.1cm}}
\pgfpathlineto{\pgfpoint{1.3cm}{-5.5cm}}
\pgfpathclose
\pgfusepath{fill}
%%
\pgfpathmoveto{\pgfpoint{-0.2cm}{-6.1cm}}
\pgfpatharc{180}{360}{0.2cm and 0.05cm}
\pgfpathclose
\pgfusepath{fill}
}
}
%    \end{macrocode}
% \end{macro}
%
% \begin{macro}{lightbulbfilled}
%
%
%%%%%%%%%%%%%%%%%%%%%%%%%%%%%%%%%%%%%%%%%%%%%%%%%%%%%%%%%%%%%%%%%%%%%%%%%%%%%%%%%%%%%%%%%%%%%%%%%%%%%%%%%%%%%%%%%%%%%%%%%%%%%%%%%%%%%%%%%%%%%%%%%%%%%%%%%%%%%%%%%%%%%
%%
%% Shape: Filled, unlit light bulb
%%
%%%%%%%%%%%%%%%%%%%%%%%%%%%%%%%%%%%%%%%%%%%%%%%%%%%%%%%%%%%%%%%%%%%%%%%%%%%%%%%%%%%%%%%%%%%%%%%%%%%%%%%%%%%%%%%%%%%%%%%%%%%%%%%%%%%%%%%%%%%%%%%%%%%%%%%%%%%%%%%%%%%%%
%
%
%
% The |lightbulbfilled| shape is a line drawing of an incandescent light bulb 
% with the bulb filled in with a gray color
%
%    \begin{macrocode}
\pgfdeclareshape{lightbulbfilled}{
\savedanchor\textpoint{%
\newdimen\ancx%
\newdimen\ancy%
\pgfextractx{\ancx}{\pgfpointorigin}%
\pgfextracty{\ancy}{\pgfpointorigin}%
\pgfpoint{\ancx}{\ancy}%
}
\anchor{center}{\pgfpointorigin}
\anchor{text}{\pgfpointorigin}
\anchorborder{%
\newdimen\@tempdimxa
\newdimen\@tempdimya
\@tempdimxa=\pgf@x
\@tempdimya=\pgf@y
\pgfpointborderrectangle{\pgfpoint{\@tempdimxa}{\@tempdimya}}
{\pgfpoint{1.5cm}{6.1cm}} %%@@TODO: fix border
}
\backgroundpath{
%%
\pgfsetlinewidth{0.25mm}
%%
\pgfsetstrokecolor{gray}
\pgfpathmoveto{\pgfpoint{-0.383cm}{-2.7cm}}
\pgfpathlineto{\pgfpoint{-0.383cm}{-1.2835cm}}
\pgfpathlineto{\pgfpoint{-1.4665cm}{0cm}}
\pgfpathlineto{\pgfpoint{-1.4665cm}{0.2cm}}
\pgfpatharc{180}{0}{0.25cm}
\pgfpatharc{0}{-180}{0.10cm}%0.3
\pgfpatharc{180}{0}{0.25cm}
\pgfpatharc{0}{-180}{0.10cm}%0.6
\pgfpatharc{180}{0}{0.25cm}
\pgfpatharc{0}{-180}{0.10cm}%0.9
\pgfpatharc{180}{0}{0.25cm}
\pgfpatharc{0}{-180}{0.10cm}%1.2
\pgfpatharc{180}{0}{0.25cm}
\pgfpatharc{0}{-180}{0.10cm}%1.5
\pgfpatharc{180}{0}{0.25cm}
\pgfpatharc{0}{-180}{0.10cm}%1.8
\pgfpatharc{180}{0}{0.25cm}
\pgfpatharc{0}{-180}{0.10cm}%2.1
\pgfpatharc{180}{0}{0.25cm}
\pgfpatharc{0}{-180}{0.10cm}%2.4
\pgfpatharc{180}{0}{0.25cm}%2.9
\pgfpathlineto{\pgfpoint{1.4665cm}{0.0cm}}
\pgfpathlineto{\pgfpoint{+0.383cm}{-1.2835cm}}
\pgfpathlineto{\pgfpoint{+0.383cm}{-2.7cm}}
\pgfusepath{stroke}
%%
\pgfsetfillcolor{gray}
\pgfpathmoveto{\pgfpoint{-1.3cm}{-5.5cm}}
\pgfpatharc{270}{90}{0.35cm}
\pgfpatharc{270}{90}{0.35cm}
\pgfpatharc{270}{90}{0.35cm}
\pgfpatharc{270}{90}{0.35cm}
\pgfpathlineto{\pgfpoint{1.3cm}{-2.7cm}}
\pgfpatharc{90}{-90}{0.35cm}
\pgfpatharc{90}{-90}{0.35cm}
\pgfpatharc{90}{-90}{0.35cm}
\pgfpatharc{90}{-90}{0.35cm}
\pgfpathclose
\pgfusepath{fill}
%%
\pgfsetstrokecolor{black}
\pgfpathmoveto{\pgfpoint{-1.3cm}{-5.5cm}}
\pgfpathlineto{\pgfpoint{1.3cm}{-4.8cm}}
%%
\pgfpathmoveto{\pgfpoint{-1.3cm}{-4.8cm}}
\pgfpathlineto{\pgfpoint{1.3cm}{-4.1cm}}
%%
\pgfpathmoveto{\pgfpoint{-1.3cm}{-4.1cm}}
\pgfpathlineto{\pgfpoint{1.3cm}{-3.4cm}}
%%
\pgfpathmoveto{\pgfpoint{-1.3cm}{-3.4cm}}
\pgfpathlineto{\pgfpoint{1.3cm}{-2.7cm}}
\pgfusepath{stroke}
%%
\pgfsetfillcolor{black}
\pgfpathmoveto{\pgfpoint{-1.3cm}{-5.5cm}}
\pgfpathlineto{\pgfpoint{-0.2cm}{-6.1cm}}
\pgfpathlineto{\pgfpoint{+0.2cm}{-6.1cm}}
\pgfpathlineto{\pgfpoint{1.3cm}{-5.5cm}}
\pgfpathclose
\pgfusepath{fill}
%%
\pgfsetfillcolor{gray}
\pgfpathmoveto{\pgfpoint{-0.2cm}{-6.1cm}}
\pgfpatharc{180}{360}{0.2cm and 0.05cm}
\pgfpathclose
\pgfusepath{fill}
}
\foregroundpath{
%% the bulb itself
\pgfsetfillcolor{white}
\pgfsetfillopacity{0.40};
\pgfpathmoveto{\pgfpoint{-1.3cm}{-2.7cm}}
\pgfpatharc{244.3207}{-64.3207}{3.0cm}
\pgfusepath{fill}
\pgfsetfillopacity{1.0};
}
}
%    \end{macrocode}
% \end{macro}
%
% \begin{macro}{lightbulbfilled}
%
%
%%%%%%%%%%%%%%%%%%%%%%%%%%%%%%%%%%%%%%%%%%%%%%%%%%%%%%%%%%%%%%%%%%%%%%%%%%%%%%%%%%%%%%%%%%%%%%%%%%%%%%%%%%%%%%%%%%%%%%%%%%%%%%%%%%%%%%%%%%%%%%%%%%%%%%%%%%%%%%%%%%%%%
%%
%% Shape: Filled, lit light bulb
%%
%%%%%%%%%%%%%%%%%%%%%%%%%%%%%%%%%%%%%%%%%%%%%%%%%%%%%%%%%%%%%%%%%%%%%%%%%%%%%%%%%%%%%%%%%%%%%%%%%%%%%%%%%%%%%%%%%%%%%%%%%%%%%%%%%%%%%%%%%%%%%%%%%%%%%%%%%%%%%%%%%%%%%
%
%
%
% The |lightbulbfilled| shape is a line drawing of an incandescent light bulb 
% with the bulb filled in with the current color
%
%    \begin{macrocode}
\pgfdeclareshape{lightbulbfilledlit}{
\savedanchor\textpoint{%
\newdimen\ancx%
\newdimen\ancy%
\pgfextractx{\ancx}{\pgfpointorigin}%
\pgfextracty{\ancy}{\pgfpointorigin}%
\pgfpoint{\ancx}{\ancy}%
}
\anchor{center}{\pgfpointorigin}
\anchor{text}{\pgfpointorigin}
\anchorborder{%
\newdimen\@tempdimxa
\newdimen\@tempdimya
\@tempdimxa=\pgf@x
\@tempdimya=\pgf@y
\pgfpointborderrectangle{\pgfpoint{\@tempdimxa}{\@tempdimya}}
{\pgfpoint{1.5cm}{6.1cm}} %%@@TODO: fix border
}
\backgroundpath{
%%
\pgfscope
\pgfsetlinewidth{0.25mm}
%%
\pgfsetstrokecolor{gray}
\pgfpathmoveto{\pgfpoint{-0.383cm}{-2.7cm}}
\pgfpathlineto{\pgfpoint{-0.383cm}{-1.2835cm}}
\pgfpathlineto{\pgfpoint{-1.4665cm}{0cm}}
\pgfpathlineto{\pgfpoint{-1.4665cm}{0.2cm}}
\pgfusepath{stroke}
%%
\pgfsetstrokecolor{gray}
\pgfpathmoveto{\pgfpoint{1.4665cm}{0.2cm}}
\pgfpathlineto{\pgfpoint{1.4665cm}{0.0cm}}
\pgfpathlineto{\pgfpoint{+0.383cm}{-1.2835cm}}
\pgfpathlineto{\pgfpoint{+0.383cm}{-2.7cm}}
\pgfusepath{stroke}
%%
\pgfsetfillcolor{gray}
\pgfpathmoveto{\pgfpoint{-1.3cm}{-5.5cm}}
\pgfpatharc{270}{90}{0.35cm}
\pgfpatharc{270}{90}{0.35cm}
\pgfpatharc{270}{90}{0.35cm}
\pgfpatharc{270}{90}{0.35cm}
\pgfpathlineto{\pgfpoint{1.3cm}{-2.7cm}}
\pgfpatharc{90}{-90}{0.35cm}
\pgfpatharc{90}{-90}{0.35cm}
\pgfpatharc{90}{-90}{0.35cm}
\pgfpatharc{90}{-90}{0.35cm}
\pgfpathclose
\pgfusepath{fill}
%%
\pgfsetstrokecolor{black}
\pgfpathmoveto{\pgfpoint{-1.3cm}{-5.5cm}}
\pgfpathlineto{\pgfpoint{1.3cm}{-4.8cm}}
%%
\pgfpathmoveto{\pgfpoint{-1.3cm}{-4.8cm}}
\pgfpathlineto{\pgfpoint{1.3cm}{-4.1cm}}
%%
\pgfpathmoveto{\pgfpoint{-1.3cm}{-4.1cm}}
\pgfpathlineto{\pgfpoint{1.3cm}{-3.4cm}}
%%
\pgfpathmoveto{\pgfpoint{-1.3cm}{-3.4cm}}
\pgfpathlineto{\pgfpoint{1.3cm}{-2.7cm}}
\pgfusepath{stroke}
%%
\pgfsetfillcolor{black}
\pgfpathmoveto{\pgfpoint{-1.3cm}{-5.5cm}}
\pgfpathlineto{\pgfpoint{-0.2cm}{-6.1cm}}
\pgfpathlineto{\pgfpoint{+0.2cm}{-6.1cm}}
\pgfpathlineto{\pgfpoint{1.3cm}{-5.5cm}}
\pgfpathclose
\pgfusepath{fill}
%%
\pgfsetfillcolor{gray}
\pgfpathmoveto{\pgfpoint{-0.2cm}{-6.1cm}}
\pgfpatharc{180}{360}{0.2cm and 0.05cm}
\pgfpathclose
\pgfusepath{fill}
\endpgfscope
}
%%
\foregroundpath{
%%
\pgfscope
\pgfpathmoveto{\pgfpoint{-1.4665cm}{0.2cm}}
\pgfpatharc{180}{0}{0.25cm}
\pgfpatharc{0}{-180}{0.10cm}%0.3
\pgfpatharc{180}{0}{0.25cm}
\pgfpatharc{0}{-180}{0.10cm}%0.6
\pgfpatharc{180}{0}{0.25cm}
\pgfpatharc{0}{-180}{0.10cm}%0.9
\pgfpatharc{180}{0}{0.25cm}
\pgfpatharc{0}{-180}{0.10cm}%1.2
\pgfpatharc{180}{0}{0.25cm}
\pgfpatharc{0}{-180}{0.10cm}%1.5
\pgfpatharc{180}{0}{0.25cm}
\pgfpatharc{0}{-180}{0.10cm}%1.8
\pgfpatharc{180}{0}{0.25cm}
\pgfpatharc{0}{-180}{0.10cm}%2.1
\pgfpatharc{180}{0}{0.25cm}
\pgfpatharc{0}{-180}{0.10cm}%2.4
\pgfpatharc{180}{0}{0.25cm}%2.9
\pgfusepath{stroke}
%%
\pgfsetfillopacity{0.60};
\pgfpathmoveto{\pgfpoint{-1.3cm}{-2.7cm}}
\pgfpatharc{244.3207}{-64.3207}{3.0cm}
\pgfusepath{fill}
\endpgfscope
%%
}
}
%    \end{macrocode}
% \end{macro}
%
% \begin{macro}{mark}
%
%
%%%%%%%%%%%%%%%%%%%%%%%%%%%%%%%%%%%%%%%%%%%%%%%%%%%%%%%%%%%%%%%%%%%%%%%%%%%%%%%%%%%%%%%%%%%%%%%%%%%%%%%%%%%%%%%%%%%%%%%%%%%%%%%%%%%%%%%%%%%%%%%%%%%%%%%%%%%%%%%%%%%%%
%%
%% Shape: Mark (X)
%%
%%%%%%%%%%%%%%%%%%%%%%%%%%%%%%%%%%%%%%%%%%%%%%%%%%%%%%%%%%%%%%%%%%%%%%%%%%%%%%%%%%%%%%%%%%%%%%%%%%%%%%%%%%%%%%%%%%%%%%%%%%%%%%%%%%%%%%%%%%%%%%%%%%%%%%%%%%%%%%%%%%%%%
%
%
%
% The |mark| shape is an "X" shape to mark a location.
%
%    \begin{macrocode}
\pgfdeclareshape{mark}{
\savedanchor\textpoint{%
\newdimen\ancx%
\newdimen\ancy%
\pgfextractx{\ancx}{\pgfpointorigin}%
\pgfextracty{\ancy}{\pgfpointorigin}%
\pgfpoint{\ancx}{\ancy}%
}
\anchor{center}{\pgfpointorigin}
\anchor{text}{\pgfpointorigin}
\anchorborder{%
\newdimen\@tempdimxa
\newdimen\@tempdimya
\@tempdimxa=\pgf@x
\@tempdimya=\pgf@y
\pgfpointborderrectangle{\pgfpoint{\@tempdimxa}{\@tempdimya}}
{\pgfpoint{0.25cm}{0.25cm}}
}
\backgroundpath{
%%
\pgfscope
\pgfsetlinewidth{0.25mm}
\pgfpathmoveto{\pgfpoint{-0.25cm}{-0.25cm}}
\pgfpathlineto{\pgfpoint{0.25cm}{0.25cm}}
\pgfusepath{stroke}
%%
\pgfpathmoveto{\pgfpoint{0.25cm}{-0.25cm}}
\pgfpathlineto{\pgfpoint{-0.25cm}{0.25cm}}
\pgfusepath{stroke}
%%
\endpgfscope
}
\foregroundpath{ }
}
%    \end{macrocode}
% \end{macro}
%
% \iffalse
%% \begin{macro}{handholding}
%%
%%
%%%%%%%%%%%%%%%%%%%%%%%%%%%%%%%%%%%%%%%%%%%%%%%%%%%%%%%%%%%%%%%%%%%%%%%%%%%%%%%%%%%%%%%%%%%%%%%%%%%%%%%%%%%%%%%%%%%%%%%%%%%%%%%%%%%%%%%%%%%%%%%%%%%%%%%%%%%%%%%%%%%%%
%%
%% Shape: Hand holding (something)
%%
%%%%%%%%%%%%%%%%%%%%%%%%%%%%%%%%%%%%%%%%%%%%%%%%%%%%%%%%%%%%%%%%%%%%%%%%%%%%%%%%%%%%%%%%%%%%%%%%%%%%%%%%%%%%%%%%%%%%%%%%%%%%%%%%%%%%%%%%%%%%%%%%%%%%%%%%%%%%%%%%%%%%%
%%
%%
%%
%% The |handholding| shape is hand holding something
%%
%%    \begin{macrocode}
%%\pgfdeclareshape{handholding}{
%%\savedanchor\textpoint{%
%%\newdimen\ancx%
%%\newdimen\ancy%
%%\pgfextractx{\ancx}{\pgfpointorigin}%
%%\pgfextracty{\ancy}{\pgfpointorigin}%
%%\pgfpoint{\ancx}{\ancy}%
%%}
%%\anchor{center}{\pgfpointorigin}
%%\anchor{text}{\pgfpointorigin}
%%\backgroundpath{
%%%
%%\pgfscope
%%\pgfsetlinewidth{0.25mm}
%%%
%%\pgftransformshift{\pgfpointscale{-1}{\pgfpointorigin}}
%%%
%%\pgfpathmoveto{\pgfpoint{-0.25cm}{-0.25cm}}
%%\pgfpathlineto{\pgfpoint{0.25cm}{0.25cm}}
%%\pgfusepath{stroke}
%%%
%%\pgfpathmoveto{\pgfpoint{0.25cm}{-0.25cm}}
%%\pgfpathlineto{\pgfpoint{-0.25cm}{0.25cm}}
%%\pgfusepath{stroke}
%%%
%%\endpgfscope
%%}
%%\foregroundpath{ }
%%}
%%    \end{macrocode}
%%
%% \end{macro}
%\fi
% \begin{macro}{cannon}
%
%
%%%%%%%%%%%%%%%%%%%%%%%%%%%%%%%%%%%%%%%%%%%%%%%%%%%%%%%%%%%%%%%%%%%%%%%%%%%%%%%%%%%%%%%%%%%%%%%%%%%%%%%%%%%%%%%%%%%%%%%%%%%%%%%%%%%%%%%%%%%%%%%%%%%%%%%%%%%%%%%%%%%%
%
% Shape: Cannon
%
%%%%%%%%%%%%%%%%%%%%%%%%%%%%%%%%%%%%%%%%%%%%%%%%%%%%%%%%%%%%%%%%%%%%%%%%%%%%%%%%%%%%%%%%%%%%%%%%%%%%%%%%%%%%%%%%%%%%%%%%%%%%%%%%%%%%%%%%%%%%%%%%%%%%%%%%%%%%%%%%%%%%
%
%
%
% The |cannon| shape is a cannon on a pedestal, used for demonstrating 
% Newtonian gravity and orbits
%
%    \begin{macrocode}
\pgfdeclareshape{cannon}{
\savedanchor\textpoint{%
\newdimen\ancx%
\newdimen\ancy%
\pgfextractx{\ancx}{\pgfpointorigin}%
\pgfextracty{\ancy}{\pgfpointorigin}%
\pgfpoint{\ancx}{\ancy}%
}
\savedanchor\eastpoint{%
\newdimen\ancx%
\newdimen\ancy%
\pgfextractx{\ancx}{\pgfpointorigin}%
\pgfextracty{\ancy}{\pgfpointorigin}%
\advance\ancx by 2.00cm
\pgfpoint{\ancx}{\ancy}%
}
\savedanchor\southpoint{%
\newdimen\ancx%
\newdimen\ancy%
\pgfextractx{\ancx}{\pgfpointorigin}%
\pgfextracty{\ancy}{\pgfpointorigin}%
\advance\ancy by -1.00cm
\pgfpoint{\ancx}{\ancy}%
}
\anchor{center}{\pgfpointorigin}
\anchor{east}{\eastpoint}
\anchor{south}{\southpoint}
\anchor{text}{\pgfpointorigin}
\anchorborder{%
\newdimen\@tempdimxa
\newdimen\@tempdimya
\@tempdimxa=\pgf@x
\@tempdimya=\pgf@y
\pgfpointborderrectangle{\pgfpoint{\@tempdimxa}{\@tempdimya}}
{\pgfpoint{2.0cm}{1.0cm}}
}
\foregroundpath{
%%
\pgfscope
\pgfsetlinewidth{0.25mm}
%% The cannon
\pgfsetfillcolor{black}
\pgfpathmoveto{\pgfpoint{-1.0cm}{0.3cm}}
\pgfpathlineto{\pgfpoint{2.0cm}{0.2cm}}
\pgfpathlineto{\pgfpoint{2.0cm}{-0.2cm}}
\pgfpathlineto{\pgfpoint{-1.0cm}{-0.3cm}}
\pgfpathclose
\pgfusepath{fill}
%% a little ball on the back of the cannon
\pgfpathcircle{\pgfpoint{-1.1cm}{0.0cm}}{0.1cm}
\pgfusepath{fill}
%% the pedestal
\pgfsetfillcolor{gray}
\pgfpathmoveto{\pgfpoint{-0.50cm}{-1.00cm}}
\pgfpathlineto{\pgfpointorigin}
\pgfpathlineto{\pgfpoint{0.50cm}{-1.00cm}}
\pgfpathclose
\pgfusepath{fill}
%%
\endpgfscope
}
}
%    \end{macrocode}
% \end{macro}
%
% \begin{macro}{eyeballleft}
%
%
%%%%%%%%%%%%%%%%%%%%%%%%%%%%%%%%%%%%%%%%%%%%%%%%%%%%%%%%%%%%%%%%%%%%%%%%%%%%%%%%%%%%%%%%%%%%%%%%%%%%%%%%%%%%%%%%%%%%%%%%%%%%%%%%%%%%%%%%%%%%%%%%%%%%%%%%%%%%%%%%%%%%
%
% Shape: Eyeball (L)
%
%%%%%%%%%%%%%%%%%%%%%%%%%%%%%%%%%%%%%%%%%%%%%%%%%%%%%%%%%%%%%%%%%%%%%%%%%%%%%%%%%%%%%%%%%%%%%%%%%%%%%%%%%%%%%%%%%%%%%%%%%%%%%%%%%%%%%%%%%%%%%%%%%%%%%%%%%%%%%%%%%%%%
%
%
%
% The |eyeballleft| shape is an eye looking to the left, used for demonstrating 
% an observer looking at an object or source of light. The color of the iris 
% can be specified by the user.
%
%    \begin{macrocode}
\pgfdeclareshape{eyeballleft}{
\savedanchor\textpoint{%
\newdimen\ancx
\newdimen\ancy
\pgfextractx{\ancx}{\pgfpointorigin}%
\pgfextracty{\ancy}{\pgfpointorigin}%
\advance\ancy by -0.25cm
\advance\ancy by -0.5\ht\pgfnodeparttextbox%
\pgfpoint{\ancx}{\ancy}%
}
\anchor{center}{\pgfpointorigin}
\anchor{text}{\textpoint}
\anchorborder{%
\newdimen\@tempdimxa
\newdimen\@tempdimya
\@tempdimxa=\pgf@x
\@tempdimya=\pgf@y
\pgfpointborderrectangle{\pgfpoint{\@tempdimxa}{\@tempdimya}}
{\pgfpoint{1.0cm}{0.30cm}}
}
\backgroundpath{
%% the iris in the user's color
\pgfpathmoveto{\pgfpointorigin}
\pgfpatharc{180}{135}{0.25cm}
\pgfpatharc{45}{-45}{0.25cm}
\pgfpatharc{225}{180}{0.25cm}
\pgfpathclose
\pgfusepath{fill}
%%
}
\foregroundpath{
\pgfsetfillcolor{black}
\pgfpathmoveto{\pgfpointorigin}
\pgfpatharc{180}{157.5}{0.25cm}
\pgfpatharc{22.5}{-22.5}{0.25cm}
\pgfpatharc{202.5}{180}{0.25cm}
\pgfpathclose
\pgfusepath{fill}
%%
\pgfsetstrokecolor{black}
\pgfsetlinewidth{0.15mm}
\pgfpathmoveto{\pgfpoint{-0.0075cm}{0}}
\pgfpatharc{180}{120}{0.25cm}
\pgfusepath{stroke}
%%
\pgfpathmoveto{\pgfpoint{-0.0075cm}{0}}
\pgfpatharc{180}{240}{0.25cm}
\pgfusepath{stroke}
%%
\pgfsetlinewidth{0.50mm}
\pgfpathmoveto{\pgfpointadd{\pgfpoint{0.2425cm}{0}}
{\pgfpointpolar{120}{0.25cm}}}
\pgfpatharc{70}{75}{3cm}
\pgfusepath{stroke}
%%
\pgfpathmoveto{\pgfpointadd{\pgfpoint{0.2425cm}{0}}
{\pgfpointpolar{120}{0.25cm}}}
\pgfpatharc{70}{60}{3cm}
\pgfusepath{stroke}
%%
\pgfpathmoveto{\pgfpointadd{\pgfpoint{0.2425cm}{0}}
{\pgfpointpolar{240}{0.25cm}}}
\pgfpatharc{290}{285}{3cm}
\pgfusepath{stroke}
%%
\pgfpathmoveto{\pgfpointadd{\pgfpoint{0.2425cm}{0}}
{\pgfpointpolar{240}{0.25cm}}}
\pgfpatharc{290}{300}{3cm}
\pgfusepath{stroke}
}
}
%    \end{macrocode}
% \end{macro}
%
% \begin{macro}{sun}
%
%
%%%%%%%%%%%%%%%%%%%%%%%%%%%%%%%%%%%%%%%%%%%%%%%%%%%%%%%%%%%%%%%%%%%%%%%%%%%%%%%%%%%%%%%%%%%%%%%%%%%%%%%%%%%%%%%%%%%%%%%%%%%%%%%%%%%%%%%%%%%%%%%%%%%%%%%%%%%%%%%%%%%%
%
% Shape: Sun
%
%%%%%%%%%%%%%%%%%%%%%%%%%%%%%%%%%%%%%%%%%%%%%%%%%%%%%%%%%%%%%%%%%%%%%%%%%%%%%%%%%%%%%%%%%%%%%%%%%%%%%%%%%%%%%%%%%%%%%%%%%%%%%%%%%%%%%%%%%%%%%%%%%%%%%%%%%%%%%%%%%%%%
%
%
%
% The |sun| shape is an dark yellow circle with lighter yellow points around 
% the edge.
%
%    \begin{macrocode}
\pgfdeclareshape{sun}{
\anchor{center}{\pgfpointorigin}
\anchor{text}{\pgfpointorigin}
\backgroundpath{
\pgfsetfillcolor{yellow!80!black}
\pgfpathmoveto{\pgfpointpolar{90}{0.5cm}}
\pgfpathlineto{\pgfpointpolar{99}{0.4cm}}
\pgfpathlineto{\pgfpointpolar{108}{0.5cm}}
\pgfpathlineto{\pgfpointpolar{117}{0.4cm}}
\pgfpathlineto{\pgfpointpolar{126}{0.5cm}}
\pgfpathlineto{\pgfpointpolar{135}{0.4cm}}
\pgfpathlineto{\pgfpointpolar{144}{0.5cm}}
\pgfpathlineto{\pgfpointpolar{153}{0.4cm}}
\pgfpathlineto{\pgfpointpolar{162}{0.5cm}}
\pgfpathlineto{\pgfpointpolar{171}{0.4cm}}
\pgfpathlineto{\pgfpointpolar{180}{0.5cm}}
\pgfpathlineto{\pgfpointpolar{189}{0.4cm}}
\pgfpathlineto{\pgfpointpolar{198}{0.5cm}}
\pgfpathlineto{\pgfpointpolar{207}{0.4cm}}
\pgfpathlineto{\pgfpointpolar{216}{0.5cm}}
\pgfpathlineto{\pgfpointpolar{225}{0.4cm}}
\pgfpathlineto{\pgfpointpolar{234}{0.5cm}}
\pgfpathlineto{\pgfpointpolar{243}{0.4cm}}
\pgfpathlineto{\pgfpointpolar{252}{0.5cm}}
\pgfpathlineto{\pgfpointpolar{261}{0.4cm}}
\pgfpathlineto{\pgfpointpolar{270}{0.5cm}}
\pgfpathlineto{\pgfpointpolar{279}{0.4cm}}
\pgfpathlineto{\pgfpointpolar{288}{0.5cm}}
\pgfpathlineto{\pgfpointpolar{297}{0.4cm}}
\pgfpathlineto{\pgfpointpolar{306}{0.5cm}}
\pgfpathlineto{\pgfpointpolar{315}{0.4cm}}
\pgfpathlineto{\pgfpointpolar{324}{0.5cm}}
\pgfpathlineto{\pgfpointpolar{333}{0.4cm}}
\pgfpathlineto{\pgfpointpolar{342}{0.5cm}}
\pgfpathlineto{\pgfpointpolar{351}{0.4cm}}
\pgfpathlineto{\pgfpointpolar{360}{0.5cm}}
\pgfpathlineto{\pgfpointpolar{  9}{0.4cm}}
\pgfpathlineto{\pgfpointpolar{ 18}{0.5cm}}
\pgfpathlineto{\pgfpointpolar{ 27}{0.4cm}}
\pgfpathlineto{\pgfpointpolar{ 36}{0.5cm}}
\pgfpathlineto{\pgfpointpolar{ 45}{0.4cm}}
\pgfpathlineto{\pgfpointpolar{ 54}{0.5cm}}
\pgfpathlineto{\pgfpointpolar{ 63}{0.4cm}}
\pgfpathlineto{\pgfpointpolar{ 72}{0.5cm}}
\pgfpathlineto{\pgfpointpolar{ 81}{0.4cm}}
\pgfpathclose
\pgfusepath{fill}
\colorlet{tikz@ball}{yellow}
\pgfpathcircle{\pgfpointorigin}{0.4cm}
\pgfshadepath{ball}{0}
}
\anchorborder{%
\newdimen\@tempdimxa
\newdimen\@tempdimya
\@tempdimxa=\pgf@x
\@tempdimya=\pgf@y
\pgfpointborderellipse{\pgfpoint{\@tempdimxa}{\@tempdimya}}
{\pgfpoint{0.50cm}{0.50cm}}
}
\savedanchor\textpoint{% manual 1035
    \pgf@x=-.5\wd\pgfnodeparttextbox % width of the box
    \pgf@y=-.5\ht\pgfnodeparttextbox % height of the box
  }%
\savedanchor\northpoint{% manual 1035
    \pgf@x=0.0pt % width of the box
    \pgf@y=0.50cm % height of the box
  }%
\savedanchor\southpoint{% manual 1035
    \pgf@x=0.0pt % width of the box
    \pgf@y=-0.50cm % height of the box
  }%
\savedanchor\eastpoint{% manual 1035
    \pgf@x=0.50cm % width of the box
    \pgf@y=0.0pt % height of the box
  }%
\savedanchor\westpoint{% manual 1035
    \pgf@x=-0.50cm % height of the box
    \pgf@y=0.0pt % width of the box
  }%
\savedanchor\northeastpoint{% manual 1035
    \pgf@x=0.3536cm % width of the box
    \pgf@y=0.3536cm % height of the box
  }%
\savedanchor\southeastpoint{% manual 1035
    \pgf@x=0.3536cm % width of the box
    \pgf@y=-0.3536cm % height of the box
  }%
\savedanchor\northwestpoint{% manual 1035
    \pgf@x=-0.3536cm % width of the box
    \pgf@y=0.3536cm % height of the box
  }%
\savedanchor\southwestpoint{% manual 1035
    \pgf@x=-0.3536cm % height of the box
    \pgf@y=-0.3536cm % width of the box
  }%
\anchor{base}{\pgfpointorigin}
\anchor{center}{\pgfpointorigin}
\anchor{text}{\textpoint}
\anchor{south}{\southpoint}
\anchor{north}{\northpoint}
\anchor{east}{\eastpoint}
\anchor{west}{\westpoint}
\anchor{south east}{\southeastpoint}
\anchor{north east}{\northeastpoint}
\anchor{south west}{\southwestpoint}
\anchor{north west}{\northwestpoint}
}
%    \end{macrocode}
% \end{macro}
%
% \begin{macro}{electron}
%
%
%%%%%%%%%%%%%%%%%%%%%%%%%%%%%%%%%%%%%%%%%%%%%%%%%%%%%%%%%%%%%%%%%%%%%%%%%%%%%%%%%%%%%%%%%%%%%%%%%%%%%%%%%%%%%%%%%%%%%%%%%%%%%%%%%%%%%%%%%%%%%%%%%%%%%%%%%%%%%%%%%%%%
%
% Shape: electron
%
%%%%%%%%%%%%%%%%%%%%%%%%%%%%%%%%%%%%%%%%%%%%%%%%%%%%%%%%%%%%%%%%%%%%%%%%%%%%%%%%%%%%%%%%%%%%%%%%%%%%%%%%%%%%%%%%%%%%%%%%%%%%%%%%%%%%%%%%%%%%%%%%%%%%%%%%%%%%%%%%%%%%
%
%
%
% The |electron| shape is an light cyan ball with a `-' in the middle.
%
%    \begin{macrocode}
\pgfdeclareshape{electron}{
\nodeparts{text}%
\savedanchor\textpoint{% manual 1035
    \pgf@x=-.5\wd\pgfnodeparttextbox % width of the box
    \pgf@y=-.5\ht\pgfnodeparttextbox % height of the box
  }%
\savedanchor\northpoint{% manual 1035
    \pgf@x=0.0pt % width of the box
    \pgf@y=0.25cm % height of the box
  }%
\savedanchor\southpoint{% manual 1035
    \pgf@x=0.0pt % width of the box
    \pgf@y=-0.25cm % height of the box
  }%
\savedanchor\eastpoint{% manual 1035
    \pgf@x=0.25cm % width of the box
    \pgf@y=0.0pt % height of the box
  }%
\savedanchor\westpoint{% manual 1035
    \pgf@x=-0.25cm % height of the box
    \pgf@y=0.0pt % width of the box
  }%
\savedanchor\northeastpoint{% manual 1035
    \pgf@x=0.25cm % width of the box
    \pgf@y=0.25cm % height of the box
  }%
\savedanchor\southeastpoint{% manual 1035
    \pgf@x=0.25cm % width of the box
    \pgf@y=-0.25cm % height of the box
  }%
\savedanchor\northwestpoint{% manual 1035
    \pgf@x=-0.25cm % width of the box
    \pgf@y=0.25cm % height of the box
  }%
\savedanchor\southwestpoint{% manual 1035
    \pgf@x=-0.25cm % height of the box
    \pgf@y=-0.25cm % width of the box
  }%
\anchor{base}{\pgfpointorigin}
\anchor{center}{\pgfpointorigin}
\anchor{text}{\textpoint}
\anchor{south}{\southpoint}
\anchor{north}{\northpoint}
\anchor{east}{\eastpoint}
\anchor{west}{\westpoint}
\anchor{south east}{\southeastpoint}
\anchor{north east}{\northeastpoint}
\anchor{south west}{\southwestpoint}
\anchor{north west}{\northwestpoint}
\anchorborder{%
\newdimen\@tempdimxa
\newdimen\@tempdimya
\@tempdimxa=\pgf@x
\@tempdimya=\pgf@y
\pgfpointborderellipse{\pgfpoint{\@tempdimxa}{\@tempdimya}}
{\pgfpoint{0.25cm}{0.25cm}}
}
\backgroundpath{
\pgfscope
\colorlet{tikz@ball}{cyan}
\pgfpathcircle{\pgfpointorigin}{0.25cm}
\pgfshadepath{ball}{0}
\pgfusepath{}
\pgftext{\color{black} \textbf{-}}
\endpgfscope
}
}
%    \end{macrocode}
% \end{macro}
%
% \begin{macro}{proton}
%
%
%%%%%%%%%%%%%%%%%%%%%%%%%%%%%%%%%%%%%%%%%%%%%%%%%%%%%%%%%%%%%%%%%%%%%%%%%%%%%%%%%%%%%%%%%%%%%%%%%%%%%%%%%%%%%%%%%%%%%%%%%%%%%%%%%%%%%%%%%%%%%%%%%%%%%%%%%%%%%%%%%%%%
%
% Shape: proton
%
%%%%%%%%%%%%%%%%%%%%%%%%%%%%%%%%%%%%%%%%%%%%%%%%%%%%%%%%%%%%%%%%%%%%%%%%%%%%%%%%%%%%%%%%%%%%%%%%%%%%%%%%%%%%%%%%%%%%%%%%%%%%%%%%%%%%%%%%%%%%%%%%%%%%%%%%%%%%%%%%%%%%
%
%
%
% The |proton| shape is a red ball with a `+' in the middle.
%
%    \begin{macrocode}
\pgfdeclareshape{proton}{
\nodeparts{text}%
\savedanchor\textpoint{% manual 1035
    \pgf@x=-.5\wd\pgfnodeparttextbox % width of the box
    \pgf@y=-.5\ht\pgfnodeparttextbox % height of the box
  }%
\savedanchor\northpoint{% manual 1035
    \pgf@x=0.0pt % width of the box
    \pgf@y=0.25cm % height of the box
  }%
\savedanchor\southpoint{% manual 1035
    \pgf@x=0.0pt % width of the box
    \pgf@y=-0.25cm % height of the box
  }%
\savedanchor\eastpoint{% manual 1035
    \pgf@x=0.25cm % width of the box
    \pgf@y=0.0pt % height of the box
  }%
\savedanchor\westpoint{% manual 1035
    \pgf@x=-0.25cm % height of the box
    \pgf@y=0.0pt % width of the box
  }%
\savedanchor\northeastpoint{% manual 1035
    \pgf@x=0.25cm % width of the box
    \pgf@y=0.25cm % height of the box
  }%
\savedanchor\southeastpoint{% manual 1035
    \pgf@x=0.25cm % width of the box
    \pgf@y=-0.25cm % height of the box
  }%
\savedanchor\northwestpoint{% manual 1035
    \pgf@x=-0.25cm % width of the box
    \pgf@y=0.25cm % height of the box
  }%
\savedanchor\southwestpoint{% manual 1035
    \pgf@x=-0.25cm % height of the box
    \pgf@y=-0.25cm % width of the box
  }%
\anchor{base}{\pgfpointorigin}
\anchor{center}{\pgfpointorigin}
\anchor{text}{\textpoint}
\anchor{south}{\southpoint}
\anchor{north}{\northpoint}
\anchor{east}{\eastpoint}
\anchor{west}{\westpoint}
\anchor{south east}{\southeastpoint}
\anchor{north east}{\northeastpoint}
\anchor{south west}{\southwestpoint}
\anchor{north west}{\northwestpoint}
\anchorborder{%
\newdimen\@tempdimxa
\newdimen\@tempdimya
\@tempdimxa=\pgf@x
\@tempdimya=\pgf@y
\pgfpointborderellipse{\pgfpoint{\@tempdimxa}{\@tempdimya}}
{\pgfpoint{0.25cm}{0.25cm}}
}
\backgroundpath{
\pgfscope
\pgfpathcircle{\pgfpointorigin}{0.25cm}
%% color tikz@ball is used internally by tikz to color the ball. 
%% We therefore have to adjust that color.
\colorlet{tikz@ball}{red!90!white}
\pgfshadepath{ball}{0}
\pgfusepath{}
\pgfsetstrokecolor{white}
\pgfsetcolor{white}
\pgftext{\color{white}\textbf{+}}
\endpgfscope
%
}
}
%    \end{macrocode}
% \end{macro}
%
%
% \subsection{Macros}
%
%
%
% These are shorthand methods of drawing a subset of the symbols above. Intended mainly for compatabiltty with old documents that used these symbols before the shapes were defined.
%
% \begin{macro}{\gfxelectron}
% The |\gfxelectron| macro draws an electron symbol (electron) at the position specified by \meta{position}. The position is not saved.
%    \begin{macrocode}
\DeclareRobustCommand{\gfxelectron}[1]%
{\node[electron] at (#1) { };}
%    \end{macrocode}
% \end{macro}
% \begin{macro}{\gfxproton}
% The |\gfxproton| macro draws a proton symbol (proton) at the position specified by \meta{position}. The position is not saved.
%
%    \begin{macrocode}
\DeclareRobustCommand{\gfxproton}[1]%
{\node[proton] at (#1) { };}
%    \end{macrocode}
% \end{macro}
% \begin{macro}{\gfxelectronrel}
% The |\gfxelectronrel| macro draws an electron symbol (electron) at the position specified by \meta{position} and then offset by \meta<offset>. The position is not saved.
%
%    \begin{macrocode}
\DeclareRobustCommand{\gfxelectronrel}[2]%
{\node[electron] at ($(#1)+(#2)$) { };}
%    \end{macrocode}
% \end{macro}
% \begin{macro}{\gfxprotonrel}
% The |\gfxprotonrel| macro draws a proton symbol (proton) at the position specified by \meta{position} and then offset by \meta<offset>. The position is not saved.
%
%    \begin{macrocode}
\DeclareRobustCommand{\gfxprotonrel}[2]%
{\node[proton] at ($(#1)+(#2)$) { };}
%    \end{macrocode}
% \end{macro}
% \begin{macro}{\gfxcapacitorH}
% The |\gfxcapacitorH| macro draws a horizontally oriented capacitor (capacitorH) at the position specified by \meta{position} and with the label \meta{label} below the shape. The coordinate of the switch is not saved, so it can't be easily connected to a circuit when using this macro.
%
%    \begin{macrocode}
\DeclareRobustCommand{\gfxcapacitorH}[2]%
{\node[draw, white, line width = 0.25mm,
 shape=capacitorH, label={[white] below:#2}] at (#1) { };}
%    \end{macrocode}
% \end{macro}
% \begin{macro}{\gfxcapacitorV}
% The |\gfxcapacitorV| macro draws a vertically oriented capacitor (capacitorV) at the position specified by \meta{position} and with the label \meta{label} below the shape. The coordinate of the switch is not saved, so it can't be easily connected to a circuit when using this macro.
%
%    \begin{macrocode}
\DeclareRobustCommand{\gfxcapacitorV}[2]%
{\node[draw, white, line width = 0.25mm, 
shape=capacitorV, label={[white] below:#2}] at (#1) {};}
%    \end{macrocode}
% \end{macro}
% \begin{macro}{\gfxcellHR}
% The |\gfxcellHR| macro draws a cell oriented horizontally with the positive side to the right (cellHR) at the position specified by \meta{position} and with the label \meta{label} below the shape. The coordinate of the switch is not saved, so it can't be easily connected to a circuit when using this macro.
%
%    \begin{macrocode}
\DeclareRobustCommand{\gfxcellHR}[2]%
{\node[draw, white, line width = 0.25mm, 
shape=cellHR, label={[white] below:#2}] at (#1) {};}
%    \end{macrocode}
% \end{macro}
% \begin{macro}{\gfxcellHL}
% The |\gfxcellHL| macro draws a cell oriented horizontally with the positive side to the left (cellHL) at the position specified by \meta{position} and with the label \meta{label} below the shape. The coordinate of the switch is not saved, so it can't be easily connected to a circuit when using this macro.
%
%    \begin{macrocode}
\DeclareRobustCommand{\gfxcellHL}[2]%
{\node[draw, white, line width = 0.25mm, 
shape=cellHL, label={[white] below:#2}] at (#1) {};}
%    \end{macrocode}
% \end{macro}
% \begin{macro}{\gfxcellVU}
% The |\gfxcellVU| macro draws a cell oriented vertically with the positive side upward (cellVU) at the position specified by \meta{position} and with the label \meta{label} below the shape. The coordinate of the switch is not saved, so it can't be easily connected to a circuit when using this macro.
%
%    \begin{macrocode}
\DeclareRobustCommand{\gfxcellVU}[2]%
{\node[draw, white, line width = 0.25mm, 
shape=cellVU, label={[white] below:#2}] at (#1) {};}
%    \end{macrocode}
% \end{macro}
% \begin{macro}{\gfxcellVD}
% The |\gfxcellVD| macro draws a cell oriented vertically with the positive side downward (cellVD) at the position specified by \meta{position} and with the label \meta{label} below the shape. The coordinate of the switch is not saved, so it can't be easily connected to a circuit when using this macro.
%
%    \begin{macrocode}
\DeclareRobustCommand{\gfxcellVD}[2]%
{\node[draw, white, line width = 0.25mm, 
shape=cellVD, label={[white] below:#2}] at (#1) {};}
%    \end{macrocode}
% \end{macro}
% \begin{macro}{\gfxresistorH}
% The |\gfxresistorH| macro draws a horizontally oriented resistor (resistorH) at the position specified by \meta{position} and with the label \meta{label} below the shape. The coordinate of the switch is not saved, so it can't be easily connected to a circuit when using this macro.
%
%    \begin{macrocode}
\DeclareRobustCommand{\gfxresistorH}[2]%
{\node[draw, white, line width = 0.25mm, 
shape=resistorH, label={[white] below:#2}] at (#1) {};}
%    \end{macrocode}
% \end{macro}
% \begin{macro}{\gfxresistorV}
% The |\gfxresistorV| macro draws a vertically oriented resistor (resistorV) at the position specified by \meta{position} and with the label \meta{label} below the shape. The coordinate of the switch is not saved, so it can't be easily connected to a circuit when using this macro.
%
%    \begin{macrocode}
\DeclareRobustCommand{\gfxresistorV}[2]%
{\node[draw, white, line width = 0.25mm, 
shape=resistorV, label={[white] below:#2}] at (#1) {};}
%    \end{macrocode}
% \end{macro}
% \begin{macro}{\gfxinductorH}
% The |\gfxinductorH| macro draws a horizontally oriented inductor (inductorH) at the position specified by \meta{position} and with the label \meta{label} below the shape. The coordinate of the switch is not saved, so it can't be easily connected to a circuit when using this macro.
%
%    \begin{macrocode}
\DeclareRobustCommand{\gfxinductorH}[2]%
{\node[draw, white, line width = 0.25mm, 
shape=inductorH, label={[white] below:#2}] at (#1) {};}
%    \end{macrocode}
% \end{macro}
% \begin{macro}{\gfxinductorV}
% The |\gfxinductorV| macro draws a vertically oriented inductor (inductorV) at the position specified by \meta{position} and with the label \meta{label} below the shape. The coordinate of the switch is not saved, so it can't be easily connected to a circuit when using this macro.
%
%    \begin{macrocode}
\DeclareRobustCommand{\gfxinductorV}[2]%
{\node[draw, white, line width = 0.25mm, 
shape=inductorV, label={[white] below:#2}] at (#1) {};}
%    \end{macrocode}
% \end{macro}
% \begin{macro}{\gfxswitchtwoH}
% The |\gfxswitchtwoH| macro draws a two position, horizontal switch (switchtwoH) at the position specified by \meta{position} and with the label \meta{label} below the shape. The coordinate of the switch is not saved, so it can't be easily connected to a circuit when using this macro.
%
%    \begin{macrocode}
\DeclareRobustCommand{\gfxswitchtwoH}[2]%
{\node[draw, white, line width = 0.25mm, 
shape=switchtwoH, label={[white] below:#2}] at (#1) {};}
%    \end{macrocode}
% \end{macro}
% \begin{macro}{\gfxswitchtwoV}
% The |\gfxswitchtwoV| macro draws a two position, vertical switch (switchtwoV) at the position specified by \meta{position} and with the label \meta{label} below the shape. The coordinate of the switch is not saved, so it can't be easily connected to a circuit when using this macro.
%
%    \begin{macrocode}
\DeclareRobustCommand{\gfxswitchtwoV}[2]%
{\node[draw, white, line width = 0.25mm, 
shape=switchtwoV, label={[white] below:#2}] at (#1) {};}
%    \end{macrocode}
% \end{macro}
% \begin{macro}{\gfxswitchthreeVDr}
% The |\gfxswitchthreeVDr| macro draws a three position, vertical switch with the input on the top and switch initially right (switchthreeVDr) at the position specified by \meta{position} and with the label \meta{label} below the shape. The coordinate of the switch is not saved, so it can't be easily connected to a circuit when using this macro.
%
%    \begin{macrocode}
\DeclareRobustCommand{\gfxswitchthreeVDr}[2]%
{\node[draw, white, line width = 0.25mm, 
shape=switchthreeVDr, label={[white] below:#2}] at (#1) {};}
%    \end{macrocode}
% \end{macro}
% \begin{macro}{\gfxswitchthreeVDl}
% The |\gfxswitchthreeVDl| macro draws a three position, vertical switch with the input on the top and switch initially left (switchthreeVDl) at the position specified by \meta{position} and with the label \meta{label} below the shape. The coordinate of the switch is not saved, so it can't be easily connected to a circuit when using this macro.
%
%    \begin{macrocode}
\DeclareRobustCommand{\gfxswitchthreeVDl}[2]%
{\node[draw, white, line width = 0.25mm, 
shape=switchthreeVDl, label={[white] below:#2}] at (#1) {};}
%    \end{macrocode}
% \end{macro}
% \begin{macro}{\gfxswitchthreeVUr}
% The |\gfxswitchthreeVUr| macro draws a three position, vertical switch with the input on the bottom and switch initially right (switchthreeVUr) at the position specified by \meta{position} and with the label \meta{label} below the shape. The coordinate of the switch is not saved, so it can't be easily connected to a circuit when using this macro.
%
%    \begin{macrocode}
\DeclareRobustCommand{\gfxswitchthreeVUr}[2]%
{\node[draw, white, line width = 0.25mm, 
shape=switchthreeVUr, label={[white] below:#2}] at (#1) {};}
%    \end{macrocode}
% \end{macro}
% \begin{macro}{\gfxswitchthreeVUl}
% The |\gfxswitchthreeVUl| macro draws a three position, vertical switch with the input on the bottom and switch initially left (switchthreeVUl) at the position specified by \meta{position} and with the label \meta{label} below the shape. The coordinate of the switch is not saved, so it can't be easily connected to a circuit when using this macro.
%
%    \begin{macrocode}
\DeclareRobustCommand{\gfxswitchthreeVUl}[2]%
{\node[draw, white, line width = 0.25mm, 
shape=switchthreeVUl, label={[white] below:#2}] at (#1) {};}
%    \end{macrocode}
% \end{macro}
% \begin{macro}{\gfxswitchthreeHRu}
% The |\gfxswitchthreeHRu| macro draws a three position, horizontal switch with the input on the left and switch initially up (switchthreeHRu) at the position specified by \meta{position} and with the label \meta{label} below the shape. The coordinate of the switch is not saved, so it can't be easily connected to a circuit when using this macro.
%
%    \begin{macrocode}
\DeclareRobustCommand{\gfxswitchthreeHRu}[2]%
{\node[draw, white, line width = 0.25mm, 
shape=switchthreeHRu, label={[white] below:#2}] at (#1) {};}
%    \end{macrocode}
% \end{macro}
% \begin{macro}{\gfxswitchthreeHRd}
% The |\gfxswitchthreeHRd| macro draws a three position, horizontal switch with the input on the left and switch initially down (switchthreeHRd) at the position specified by \meta{position} and with the label \meta{label} below the shape. The coordinate of the switch is not saved, so it can't be easily connected to a circuit when using this macro.
%
%    \begin{macrocode}
\DeclareRobustCommand{\gfxswitchthreeHRd}[2]%
{\node[draw, white, line width = 0.25mm, 
shape=switchthreeHRd, label={[white] below:#2}] at (#1) {};}
%    \end{macrocode}
% \end{macro}
% \begin{macro}{\gfxswitchthreeHLu}
% The |\gfxswitchthreeHLu| macro draws a three position, horizontal switch with the input on the right and switch initially up (switchthreeHLu) at the position specified by \meta{position} and with the label \meta{label} below the shape. The coordinate of the switch is not saved, so it can't be easily connected to a circuit when using this macro.
%
%    \begin{macrocode}
\DeclareRobustCommand{\gfxswitchthreeHLu}[2]%
{\node[draw, white, line width = 0.25mm, 
shape=switchthreeHLu, label={[white] below:#2}] at (#1) {};}
%    \end{macrocode}
% \end{macro}
% \begin{macro}{\gfxswitchthreeHLd}
% The |\gfxswitchthreeHLd| macro draws a three position, horizontal switch with the input on the right and switch initially down (switchthreeHLd) at the position specified by \meta{position} and with the label \meta{label} below the shape. The coordinate of the switch is not saved, so it can't be easily connected to a circuit when using this macro.
%
%    \begin{macrocode}
\DeclareRobustCommand{\gfxswitchthreeHLd}[2]%
{\node[draw, white, line width = 0.25mm, 
shape=switchthreeHLd, label={[white] below:#2}] at (#1) {};}
%    \end{macrocode}
% \end{macro}
% \begin{macro}{\gfxvoltmeter}
% The |\gfxvoltmeter| macro draws an voltmeter (voltmeter) shape at the position specified by \meta{position} and with the label \meta{label} below the shape. The coordinate of the voltmeter is not saved, so it can't be easily connected to a circuit when using this macro.
%
%    \begin{macrocode}
\DeclareRobustCommand{\gfxvoltmeter}[2]%
{\node[white,line width=0.25mm,voltmeter,
 label={[white] below:#2}] at (#1) {};}
%    \end{macrocode}
% \end{macro}
% \begin{macro}{\gfxammeter}
% The |\gfxammeter| macro draws an ammeter (ammeter) shape at the position specified by \meta{position} and with the label \meta{label} below the shape. The coordinate of the ammeter is not saved, so it can't be easily connected to a circuit when using this macro.
%
%    \begin{macrocode}
\DeclareRobustCommand{\gfxammeter}[2]%
{\node[white,line width=0.25mm,ammmeter, 
label={[white] below:#2}] at (#1) {};}
%    \end{macrocode}
% \end{macro}
% \begin{macro}{\gfxdcsource}
% The |\gfxdcsource| macro draws an DC source (DCsource) shape at the position specified by \meta{position} and with the label \meta{label} below the shape. The coordinate of the DC source is not saved, so it can't be easily connected to a circuit when using this macro.
%
%    \begin{macrocode}
\DeclareRobustCommand{\gfxdcsource}[2]%
{\node[draw, white, line width = 0.25mm, 
shape=DCsource, label={[white] below:#2}] at (#1) {};}
%    \end{macrocode}
% \end{macro}
% \begin{macro}{\gfxacsource}
% The |\gfxacsource| macro draws an AC source (ACsource) shape at the position specified by \meta{position} and with the label \meta{label} below the shape. The coordinate of the AC source is not saved, so it can't be easily connected to a circuit when using this macro.
%
%    \begin{macrocode}
\DeclareRobustCommand{\gfxacsource}[2]%
{\node[draw, white, line width = 0.25mm, 
shape=ACsource, label={[white] below:#2}] at (#1) {};}
%    \end{macrocode}
% \end{macro}
% \begin{macro}{\gfxvectout}
% The |\gfxvectin| macro draws ``outward'' vector at the location specified by \meta{position}.
%
%    \begin{macrocode}
\DeclareRobustCommand{\gfxvectout}[1]%
{\node[yellow, line width = 0.25mm, 
shape=vectIn] at (#1) {};}
%    \end{macrocode}
% \end{macro}
% \begin{macro}{\gfxvectin}
% The |\gfxvectin| macro draws ``inward'' vector at the location specified by \meta{position}.
%
%    \begin{macrocode}
\DeclareRobustCommand{\gfxvectin}[1]%
{\node[yellow, line width = 0.25mm, 
shape=vectOut] at (#1) {}}
%    \end{macrocode}
% \end{macro}
% \begin{macro}{\gfxtransformerH}
% The |\gfxtransformerH| macro draws a horizontal transformer (transformerH) shape at the position specified by \meta{position} and with the label \meta{label} below the shape. The coordinate of the transformer is not saved, so it can't be easily connected to a circuit when using this macro.
%
%    \begin{macrocode}
\DeclareRobustCommand{\gfxtransformerH}[2]%
{\node[draw, white, line width = 
0.25mm, shape=transformerH, label={[white] below:#2}] at (#1) {};}
%    \end{macrocode}
% \end{macro}
% \Finale
\makeatother

